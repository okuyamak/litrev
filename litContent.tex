\documentclass{article}
%\usepackage{pdflscape}
%\usepackage{afterpage}
%\usepackage{capt-of}
\usepackage{booktabs}
%\usepackage{tabulary}
%\usepackage{rotating}	
%\newcommand{\tabitem}{~~\llap{\textbullet}~~}
\usepackage{longtable}
\usepackage{array} % to call magic m
\usepackage{enumitem} % to remove vertical space for itemize
\usepackage{geometry}
\geometry{a4paper, portrait, margin=1in}
\newcommand{\tabitem}{~~\llap{\textbullet}~~}

\title{Litterature review by topic}
\author{Kenta Okuyama}

\begin{document}
\pagenumbering{gobble}
\maketitle
\tableofcontents
\newpage
\pagenumbering{arabic}


%%%%%%%%%%%%%%%%%%%%%%%%%%%%%%%%%%%%%%%%%%%%
%%%%%%%%%%%%%%%%%%%%%%%%%%%%%%%%%%%%%%%%%%%%
\section{Neighborhood environments and obesity}
\subsection{References about physical environment}

\begin{itemize}
		\item {\bf Pearce (2009) -- -- Cross-sectional study}
				\begin{description}
						\item[Objectives] \mbox{}\par
							\begin{itemize}
								\item 
							\end{itemize}
						\item[Settings and Subjects] \mbox{}\par
							\begin{itemize}
								\item 
							\end{itemize}
						\item[Outcome]\mbox{}\par
							\begin{itemize}
								\item 
							\end{itemize}
						\item[Exposure]\mbox{}\par
							\begin{itemize}
								\item 
							\end{itemize}
						\item[Confounders]\mbox{}\par
							\begin{itemize}
								\item 
							\end{itemize}
						\item[Statistical analysis]\mbox{}\par
							\begin{itemize}
								\item 
							\end{itemize}
						\item[Findings]\mbox{}\par
							\begin{itemize}
								\item 
							\end{itemize}
						\item[Strengths]\mbox{}\par
							\begin{itemize}
								\item 
							\end{itemize}
						\item[Limitations]\mbox{}\par
							\begin{itemize}
								\item 
							\end{itemize}
						\item[Significance]\mbox{}\par
							\begin{itemize}
								\item Examining the fast fod outlets and diet-related health based on previous finding: fast food outlets were stratified by neighborhood deprivation.
							\end{itemize}
						\item[Conclusion]\mbox{}\par
							\begin{itemize}
								\item 
							\end{itemize}
						\item[Takeaways] \mbox{}\par
							\begin{itemize}
								\item[$\clubsuit$] People in low SES and living in high deprivided neighborhood tend to have worse diet-related health outcomes.
								\item[$\clubsuit$] In the US, higily deprived areas tend to have more fast food outlets.
								\item[$\clubsuit$] In the US, fast food outlets are associated with obesity, but it is mixed in other regions.
								\item[$\clubsuit$] High levels of residential segregation can be a reason that neighobrhoods in the US may influence individual-level health outcomes to a greater extent.  
								\item[$\clubsuit$] Targeted consumers in specific geographical localities, differences in land use planning strategies, and variations in resident's abilities to influence political decision making are caused by residential segregation.
								\item[$\clubsuit$] Future studies should include both health and unhealthy food outlets. 
							\end{itemize}
			    \end{description}
%%%%%%%%%%%%%%%%%%%%%%%%%%%%%%%%%%%%%%%%%%%%

%%%%%%%%%%%%%%%%%%%%%%%%%%%%%%%%%%%%%%%%%%%%
    \item {\bf Barrientos (2017)  -- 45-84 years old -- Prospective cohort}
		\begin{description}
			\item[Objectives] \mbox{}\par
				\begin{itemize}
					\item 
				\end{itemize}
			\item[Settings and Subjects] \mbox{}\par
				\begin{itemize}
					\item 6 US urban areas
				\end{itemize}
			\item[Outcome] \mbox{}\par
				\begin{itemize}
					\item BMI
				\end{itemize}
			\item[Exposure] \mbox{}\par
				\begin{itemize}
					\item Perceptive: Healthy food availability, walkablle environment. -- survey was conducted to different sample and aggregated for 1 mile and used as proxy for this study samples. 
					\item Objective: Density of supaermarkets and fruit-and-vegetable markets, recreational resouces within 1 mile form each participant.
				\end{itemize}
			\item[Confounders] \mbox{}\par
				\begin{itemize}
					\item Time-invariant: age, sex, race, education, duration of residence. 
					\item Time-varying: marital status, income, cancer diagnosis -- missing information was inputed from the closest year's examination.  
				\end{itemize}
			\item[Statistical analysis] \mbox{}\par
				\begin{itemize}
					\item Within-person change for environment vs within-persion change in BMI was estimated by fixed-effects models. 
					\item Model 1 included each measure separately -- food and physical activity (perceptive and objective).
					\item Model 2 included  percetive and objective simultaneously by food and physical activity. 
					\item Model 3 included all at once.
					\item Model 4 included z-score aggregated by food and physical activity simultaneously.
					\item In regression, neighborhood measures were transformed into sd. 
					\item Sensitivity analysis was made for non-movers during the study period
				\end{itemize}	
			\item[Findings]\mbox{}\par
				\begin{itemize}
					\item 
				\end{itemize}
			\item[Strengths]\mbox{}\par
				\begin{itemize}
					\item 
				\end{itemize}
			\item[Limitations]\mbox{}\par
				\begin{itemize}
					\item 
				\end{itemize}
			\item[Significance]\mbox{}\par 
				\begin{itemize}
					\item First longitudinal study accounting for changes in envrionments in both perceptive and objective measures again changes in BMI. 
				\end{itemize}
			\item[Conclusion]\mbox{}\par 
				\begin{itemize}
					\item Favorable changes in both food and physical activity environment  was associated with BMI reductions in obese and overweight persons.
				\end{itemize}
    			\item[Takeaways] \mbox{}\par
    				\begin{itemize}
    					\item[$\clubsuit$] In recent years, neighborhood study moved from a general framework to identifying the specific mechanism -- which neighborhood influence health.
    					\item[$\clubsuit$] Yet longitudinal evidences are still scarce.
    					\item[$\clubsuit$] The few studies investigated whether changes in food availability are related to changes in diet and BMI. Few studies for physical activity environment -- only 1 study found that improve of recreational facilities access was associated with decrease in BMI. 
				\end{itemize}
    		\end{description}
%%%%%%%%%%%%%%%%%%%%%%%%%%%%%%%%%%%%%%%%%%%%

%%%%%%%%%%%%%%%%%%%%%%%%%%%%%%%%%%%%%%%%%%%%
	\item {\bf Mason (2018) -- 40-69 years old -- Cross sectional study}
		\begin{description}
			\item[Objectives]\mbox{}\par 
				22 UK Biobank assessment centers
			\item[Settings and Subjects]\mbox{}\par 
				\begin{itemize}
					\item 22 UK Biobank assessment centers
				\end{itemize}
			\item[Outcome]\mbox{}\par
				\begin{itemize}
					\item Waist circumference, BMI and body fat percentage (BIA) -- centred around the mean  	
				\end{itemize}
			\item[Exposure] \mbox{}\par
				\begin{itemize}
					\item Physical activity facilities density within 1000m. 
					\item Distance to the nearest fastfood outlets categorized as 500, 500-999, 1000-1999, 2000m.
				\end{itemize}
			\item[Confounders] \mbox{}\par
				\begin{itemize} 
					\item Age, sex, ethnicity, education, income, employment, deprivation, urbanicity, residential density.
				\end{itemize}
			\item[Statistical analysis] \mbox{}\par
				\begin{itemize}
					\item Multilevel multiple linear regression with random intercepts and coefficients, accounting for the nesting of individuals within assessment centers.
					\item Final model controled for all covariates and non-exposure environment.
				\end{itemize}
			\item[Findings]\mbox{}\par
				\begin{itemize}
					\item 
				\end{itemize}
			\item[Strengths]\mbox{}\par
				\begin{itemize}
					\item
				\end{itemize}
			\item[Limitations]\mbox{}\par
				\begin{itemize}
					\item 
				\end{itemize}
			\item[Significance] \mbox{}\par
				\begin{itemize}
					\item Large national dataset which made it possible for sensitivity analysis to strengthen the robustness. 
					\item Most previous studies focused on particular areas. 
					\item Comprehensive confounding information from reliable dataset. 
					\item Focused on commercial physical activity facilities as they are modifiable via regulatory.
				\end{itemize}
    			\item[Conclusion] \mbox{}\par
						\begin{itemize}
							\item Physical activity facilities density was associated with lower adiposity. 
							\item Fast food outlets proximity was also associated with lower adiposity. 
							\item The association remained the same after stratified by sex, and income groups.
						\end{itemize}
					\item[Takeaways] \mbox{}\par
    				\begin{itemize}
    					\item[$\clubsuit$] Many research on access to fast-food outlets and obesity in the US, yet relatively little research on formal facilities for recreational physical activity.
    					\item[$\clubsuit$] Many research on walkability which focus more on urban designs than receattional facilities.
    					\item[$\clubsuit$] As food and physical activity environments are associated with one another, they are likely to confound the association with adiposity each other -- they should be included as one of the confounders. 
    				\end{itemize}	
		\end{description}
%%%%%%%%%%%%%%%%%%%%%%%%%%%%%%%%%%%%%%%%%%%%

%%%%%%%%%%%%%%%%%%%%%%%%%%%%%%%%%%%%%%%%%%%%
	\item{\bf Hirsch (2014) -- 45-84 years old -- Prospective cohort} 
		\begin{description}
			\item[Objectives] \mbox{}\par
				\begin{itemize}
					\item 
				\end{itemize}
			\item[Settings and Subjects] \mbox{}\par
				\begin{itemize}
					\item 6 US urban cities
				\end{itemize}
			\item[Outcome]\mbox{}\par
				\begin{itemize}
					\item Self report walking time for recreation and transportation.
				\end{itemize}
			\item[Exposure]\mbox{}\par
				\begin{itemize}
					\item Population density, retail area, residential area, social destinations, walking destinations within 1 mile from each resident, distance to bus stops, and street network ratio.
				\end{itemize}
			\item[Confounders] \mbox{}\par
				\begin{itemize}
					\item Time-invariate: age, sex, race, and education. 
					\item Time-varying: income, employment, marital status, car ownership, self-rated health, and arthritis. 
				\end{itemize}
			\item[Statistical analysis] \mbox{}\par
				\begin{itemize}
					\item Linear mixed models to estimate the associations of changes in the built environment and changes in walking (transportatin and recreation). 
					\item Built environment measures were separetely included in the models to avoid multicolinearlity.
				\end{itemize}
			\item[Findings]\mbox{}\par
				\begin{itemize}
					\item 
				\end{itemize}
			\item[Strengths]\mbox{}\par
				\begin{itemize}
					\item
				\end{itemize}
			\item[Limitations]\mbox{}\par
				\begin{itemize}
					\item 
				\end{itemize}
			\item[Significance] \mbox{}\par
				\begin{itemize}
					\item First study examined the time-varying GIS-based built environment measures and changes in walking.
				\end{itemize}
			\item[Conclusion] \mbox{}\par
				\begin{itemize}
					\item Higher residential use and distance to busses were associated with a slightly increase in walking. 
					\item Incrases in te number of social destinations, walking destinations, and street connectivity were associated with greater increases in wallking for transportation. 
					\item Higher baseline levels of retail and walking destinations were associated with grater increases in leasure walking, but no changes in built environment measures were associated with leisure walking.
				\end{itemize}
			\item[Takeaways] \mbox{}\par
				\begin{itemize}
					\item[$\clubsuit$] Samples completed at least 2 examinations including baseline, complete information on walking outcomes or built environment at examinations were included.
					\item[$\clubsuit$] Several movers vs non-movers studies found that residential relocation to more walkable environment resulted in increases in physical activity -- yet, these might contain unobservable preferences related to both choice of residential location and behavior.
					\item[$\clubsuit$] Future studies should aim to identify what types of changes are necessary to increase physical activity levels -- potential thresholds
				\end{itemize}
		\end{description}
%%%%%%%%%%%%%%%%%%%%%%%%%%%%%%%%%%%%%%%%%%%%

%%%%%%%%%%%%%%%%%%%%%%%%%%%%%%%%%%%%%%%%%%%%
\item{\bf Sallis(1998) -- 7 studies  -- Review} 
		\begin{description}
			\item[Objectives] NA 
			\item[Settings and Subjects] NA 
			\item[Outcome] NA 
			\item[Exposure] NA 
			\item[Confounders] NA 
			\item[Statistical analysis] NA 
			\item[Significance] First study reviewed environmental and policy intervention to promote physical activity. Proposed model on how environmental and policy interventions are implemented.  
			\item[Conclusion] Implementing environment and political level interventions is difficult due to lack of conceptual models and difficulties of evaluation. Further research is needed, and multiple sectors should collaborate more. 
			\item[Takeaways] \mbox{}\par
				\begin{itemize}
					\item[$\clubsuit$] As of late 1900s, environmental and policy interventions were infrequently applied for the control of chronic diseases. 
					\item[$\clubsuit$] Ecological and social-ecological models of human behavior have evolved in the fields of sociology, psychology, education, and public health. 
					\item[$\clubsuit$] The concept of ecological model which describes the levels of influence on behaviors are developed by McLeroy. 
				\end{itemize} 
		\end{description}
%%%%%%%%%%%%%%%%%%%%%%%%%%%%%%%%%%%%%%%%%%%%

%%%%%%%%%%%%%%%%%%%%%%%%%%%%%%%%%%%%%%%%%%%%
\item{\bf Ng(2014) -- 183 countries -- Systematic analysis} 
		\begin{description}
			\item[Setting] Worldwide 
			\item[Outcome] Prevalence of obesity (Overweight:25-30, Obesity:>=30) 
			\item[Exposure] NA 
			\item[Confounders] NA 
			\item[Statistical analysis] Mixed effects linear regression to correct for bias in self-reports. Spatiotemploral Gaussian process regression model to estimate prevalence with 95 percent uncertainty intervals. 
			\item[Significance] Up-to-date information for global obesity trends. It will be important for decision making on what action is needed and where progress is. 
			\item[Conclusion] Especially in low-income and middle-income countries, urgent intervention is needed to modify obesogenic environment. 
			\item[Takeaways] \mbox{}\par
				\begin{itemize}
					\item[$\clubsuit$] Although health risk of obesity is established and obesity prevalence has been increasing worldwide, no national sucess stories have been reported in the past 33 years. 
					\item[$\clubsuit$] The rising prevalence of overweight and obesity in several countries has been described as a global pandemic.  
					\item[$\clubsuit$] In 2010, overweight and obesity were estimated to cause 3-4 million deaths, 4 perent years of life lost, and 4 percent of disability-adjusted life-years worldwide. 
					\item[$\clubsuit$] Prevalence of obesity and overweight has incraesed substantially in the past 30 years. The pattern of increase differ by regions, and it has been attenuated in developed countries in the past 8 years.
					\item[$\clubsuit$] The increase of obesity prevalence can be explained by changes in energy intake such as high fat and calorie diet, decrease in energy expenditure such as decrease in physical activity, and chages in the gut microbiome.
					\item[$\clubsuit$] Most deaths atributable to overweight and obesity are cardiovascular deaths. Only 31 percent of he coronary heart disease risk and 8 percent of the stroke mortality risk associated with obesity is mediated through raised blood pressure and cholesterol.
				\end{itemize} 
		\end{description}
%%%%%%%%%%%%%%%%%%%%%%%%%%%%%%%%%%%%%%%%%%%%
		
%%%%%%%%%%%%%%%%%%%%%%%%%%%%%%%%%%%%%%%%%%%%
\item{\bf Kawakami(2011) -- 35-80 years  -- Prospective cohort} 
		\begin{description}
			\item[Objectives] \mbox{}\par
				\begin{itemize}
					\item 
				\end{itemize}
			\item[Settings and Subjects] \mbox{}\par
				\begin{itemize}
					\item Sweden
				\end{itemize}
			\item[Outcome]\mbox{}\par
				\begin{itemize}
					\item Age-standardised incidence proportions (proportions of subjects who became cases among those who entered the study time interval) for men and women separetely. 
				\end{itemize}
			\item[Exposure] \mbox{}\par
				\begin{itemize}
					\item Fastfood restaurants, bars/pubs, physical activity facilities, healthcare facilities by neighborhood count, individual buffer count, and distance. 
				\end{itemize}
			\item[Confounders] \mbox{}\par
				\begin{itemize}
					\item Age, income, neighborhood deprivation index. 
				\end{itemize}
			\item[Statistical analysis] \mbox{}\par
				\begin{itemize}
					\item Multilevel logistic regression for incidence propotions of CHD as an outcome. 
					\item Model 1. CHD vs ne. 
					\item Model 2. CHD vs ne + nedep. 
					\item Model 3. CHD vs ne + nedep + age + income.  
				\end{itemize}
			\item[Findings]\mbox{}\par
				\begin{itemize}
					\item 
				\end{itemize}
			\item[Strengths]\mbox{}\par
				\begin{itemize}
					\item
				\end{itemize}
			\item[Limitations]\mbox{}\par
				\begin{itemize}
					\item
				\end{itemize}
			\item[Significance]\mbox{}\par 
				\begin{itemize}
					\item First study conducted multilevel investigation to examine the longitudinal individual-level association between CHD and neighborhood availability of potentially health-promoting and health-damaging goods. 
				\end{itemize}
			\item[Conclusion] \mbox{}\par
				\begin{itemize}
					\item New Zealand
				\end{itemize}
			\item[Takeaways] \mbox{}\par
				\begin{itemize}
					\item[$\clubsuit$] Several studies found that neighborhood SES affecs cardiovascular health over individual-level SES.
					\item[$\clubsuit$] Living in deprived neighborhood tend to have limited access to healthy food resources, and it would lead to increase the risk of CHD. However, no clear pattern was found between the availability of different types of resources and level of neighborood deprivation. 
					\item[$\clubsuit$] There was a week association between neighborhood availability between CHD. There was an unexpected direction association between physical activity and healthcare facilities and CHD. 
					\item[$\clubsuit$] The association between neighborhood deprivation and CHD are well established, but the causal pathway is largely unknown. This study aimed to identify whether neighborhood health-damageing/promoting facililites lie in the causal pathway. The findings did not give expected results. Neighborhood deprivation is equal to neighborhood SES? Is the assocaition between neighborhood deprivation/ses and obesity also established? 
				\end{itemize} 
		\end{description}
%%%%%%%%%%%%%%%%%%%%%%%%%%%%%%%%%%%%%%%%%%%%

%%%%%%%%%%%%%%%%%%%%%%%%%%%%%%%%%%%%%%%%%%%%
\item{\bf Hamano(2017) -- 0-14 years  -- Prospective cohort} 
		\begin{description}
			\item[Objectives] \mbox{}\par
				\begin{itemize}
					\item
				\end{itemize}
			\item[Settings and Subjects] \mbox{}\par
				\begin{itemize}
					\item Sweden
				\end{itemize}
			\item[Outcome] \mbox{}\par
				\begin{itemize}
					\item Obesity diagnosed by ICD-10. 
				\end{itemize}
			\item[Exposure] \mbox{}\par
				\begin{itemize}
					\item Fastfood restaurants, within SAMS and 1000m buffer.
				\end{itemize}
			\item[Confounders] \mbox{}\par
				\begin{itemize}
					\item Age, maternal marital status,family income, parent education level, parent birth place, maternal urban/rural status, moving status, maternal age at hildbirth, parent hospitalisation, family history of obesity. 
				\end{itemize}
			\item[Statistical analysis] \mbox{}\par
				\begin{itemize}
					\item Multilevel logistic regression for the cumulative rate of obesity. 
				\end{itemize}
			\item[Findings]\mbox{}\par
				\begin{itemize}
					\item 
				\end{itemize}
			\item[Strengths]\mbox{}\par
				\begin{itemize}
					\item
				\end{itemize}
			\item[Limitations]\mbox{}\par
				\begin{itemize}
					\item
				\end{itemize}
			\item[Significance] \mbox{}\par
				\begin{itemize}
					\item Multilevel analysis by controlling both individual and neighborhood SES with an large follow-up sample.
				\end{itemize}
			\item[Conclusion] \mbox{}\par
				\begin{itemize}
					\item Fast food outlets were associated with childhood obesity after adjusting for individal and neighborhood SES.
				\end{itemize}
			\item[Takeaways] \mbox{}\par
				\begin{itemize}
					\item[$\clubsuit$] Further studies should be done by taking other physical environmental features into accont.
				\end{itemize} 
		\end{description}
%%%%%%%%%%%%%%%%%%%%%%%%%%%%%%%%%%%%%%%%%%%%

%%%%%%%%%%%%%%%%%%%%%%%%%%%%%%%%%%%%%%%%%%%%
\item{\bf Sundquist(2014) -- Unknown  -- Prospective cohort} 
		\begin{description}
			\item[Objectives] \mbox{}\par
				\begin{itemize}
					\item 
				\end{itemize} 
			\item[Setting] \mbox{}\par
				\begin{itemize}
					\item Sweden - Stockholm 
				\end{itemize} 
			\item[Outcome]\mbox{}\par
				\begin{itemize}
					\item Clinically diagnosed type 2 diabetes.
				\end{itemize} 
			\item[Exposure] \mbox{}\par
				\begin{itemize}
					\item Neighborhood walkability index.  
				\end{itemize} 
			\item[Confounders] \mbox{}\par
				\begin{itemize}
					\item Age, gender, income, education, NDI. 
				\end{itemize} 
			\item[Statistical analysis] \mbox{}\par
				\begin{itemize}
					\item Multilevel logistic regression with individuals nested within their neighborhood. 
				\end{itemize} 
			\item[Findings]\mbox{}\par
				\begin{itemize}
					\item 
				\end{itemize}
			\item[Strengths]\mbox{}\par
				\begin{itemize}
					\item
				\end{itemize}
			\item[Limitations]\mbox{}\par
				\begin{itemize}
					\item
				\end{itemize}
			\item[Significance] \mbox{}\par
				\begin{itemize}
					\item This was one of the few studies examine the objectively measured walkability and health oucome, i.e., incidence of diabetes in a large cohort. 
				\end{itemize} 
			\item[Conclusion] \mbox{}\par
				\begin{itemize}
					\item There was a significant association between walkable enviornment and incidence of type 2 diabetes after adjusting for neighborhood deprivation. 
					\item However the association did not remain after further adjusting for individual socio-demographic characteristics. 
					\item Future research should consider other potential risk factors for diabetes, such as traffic noise or air pollution which may come along with walkable environment. 
				\end{itemize} 
			\item[Takeaways] \mbox{}\par
				\begin{itemize}
					\item[$\clubsuit$] There is a need to examine whether individual and neighborhood socioeconomic characteristics may modify the association between the built environment and health related behaviors (Lovasi 2009). Because socioeconomically disadvantaged individuals may be less likely to benefit out from neighborhood envrionment. 
					\item[$\clubsuit$] No significant interaction was found between walkability and individual SES, or walkability and neighborhood deprivation.
					\item[$\clubsuit$] Several studies found association between walkability and physical activity after adjusting for individal and neighborhood socio-demographic factors (Sundquist 2011). 
				\end{itemize} 
		\end{description}
%%%%%%%%%%%%%%%%%%%%%%%%%%%%%%%%%%%%%%%%%%%%

%%%%%%%%%%%%%%%%%%%%%%%%%%%%%%%%%%%%%%%%%%%%
\item{\bf Sundquist(2011) -- 20-65 years  -- Cross-sectional study}
		\begin{description}
			\item[Objectives] \mbox{}\par
				\begin{itemize}
					\item
				\end{itemize}
			\item[Setting]\mbox{}\par
				\begin{itemize}
					\item Sweden - Stockholm 
				\end{itemize}
			\item[Outcome] \mbox{}\par
				\begin{itemize}
					\item Physical activity: MVPA min/day, Time in 10-minutes bouts of MVPA min/day, Walking for AT min/day, Walking for leisure min/day. 
				\end{itemize}
			\item[Exposure] \mbox{}\par
				\begin{itemize}
					\item Neighborhood walkability index.  
				\end{itemize}
			\item[Confounders] \mbox{}\par
				\begin{itemize}
					\item Age, gender, marital status, family income, neighborhood income. 
				\end{itemize}
			\item[Statistical analysis] \mbox{}\par
				\begin{itemize}
					\item Multilevel linear regression with individuals at the first level and neighborhoods at the second level. 
					\item Model 1: PA vs walkbility. 
					\item Model 2: PA vs walkability + individal characteristics + neiborhood income. 
				\end{itemize}
			\item[Findings]\mbox{}\par
				\begin{itemize}
					\item 
				\end{itemize}
			\item[Strengths]\mbox{}\par
				\begin{itemize}
					\item
				\end{itemize}
			\item[Limitations]\mbox{}\par
				\begin{itemize}
					\item
				\end{itemize}
			\item[Significance] \mbox{}\par
				\begin{itemize}
					\item First study in Sweden to examine objectively measured PA and obejectively measure walkability.  
				\end{itemize}
			\item[Conclusion] \mbox{}\par
				\begin{itemize}
					\item There is a significant association between walkability and physical activity in Swedish adults. 
				\end{itemize}
			\item[Takeaways] \mbox{}\par
				\begin{itemize}
					\item[$\clubsuit$] Interaction between walkability and neighborhood level SES wre not found.
					\item[$\clubsuit$] Australia had a significant interaction between SES and walkability (Leslie, 2007), while no interaction was found in Sweden and Belgium. It might be because of low SES inequalities in these countries.  
				\end{itemize} 
		\end{description}
%%%%%%%%%%%%%%%%%%%%%%%%%%%%%%%%%%%%%%%%%%%%

%%%%%%%%%%%%%%%%%%%%%%%%%%%%%%%%%%%%%%%%%%%%
\item{\bf Gortmaker(2011) -- NA  -- Review} 
		\begin{description}
			\item[Objectives] 
				\begin{itemize}
					\item 
				\end{itemize} 
			\item[Setting] 
				\begin{itemize}
					\item Australia, US 
				\end{itemize} 
			\item[Outcome] 
				\begin{itemize}
					\item Effect of policy level intervention 
				\end{itemize} 
			\item[Exposure] NA
			\item[Confounders] NA
			\item[Statistical analysis] NA
			\item[Significance] Identify several cost-effective policies that gonvernment priotize for obesity prevention. 
			\item[Conclusion] A rapid increse of efforts for cost-effectiveness analysies of programmes and polices for obesity prevention is needed. 
			\item[Takeaways] \mbox{}\par
				\begin{itemize}
					\item[$\clubsuit$] Important causes for obesity is identified, which is a result of changes in the global food system - the movement from individual to mas preparation, producing more highly processed food.   
					\item[$\clubsuit$] Other factors: national wealth, government policy, cultural norms, built environment, genetic and epigenetic mechanisms, biological bases for food preferences, biological mechanisms that regulate motivation for physical activity amplify or attenumate the effect of thoes causes (nutrition change), and all infulence growth of the epidemic. 
					\item[$\clubsuit$] Most countries do not have enough monitoring data for population physical activity and diet pattern, and obesity prevalence, and it stands as barrier to set an appropriate goal and also assess progress. 
					\item[$\clubsuit$] Obesity could yeild in not only lowering future life expectancy but also increasing short-term and long-term healthcare spendings.
					\item[$\clubsuit$] Body weight response to a change of energy balance is slow. A small but chronic daily energy imbalance gap has caused the continuing weight gain seen in most countries.
					\item[$\clubsuit$] Population intervention for obesity should have effect on equity, acceptability to stakeholders, feasibility of implementation, affordability and sustainability to put policy decision forward. 
					\item[$\clubsuit$] Goverment is the main actor for obesity prevention.Obesity mainly burdens the health system, but various sectors, i.e. finance, eudcation, agriculture, transportation and urban planning have the greatest impact on creating environment conducive to prevention. 
				\end{itemize} 
		\end{description}
%%%%%%%%%%%%%%%%%%%%%%%%%%%%%%%%%%%%%%%%%%%%

%%%%%%%%%%%%%%%%%%%%%%%%%%%%%%%%%%%%%%%%%%%%
\item{\bf Do(2018) -- 45-84 years  -- Longitudinal study}
		\begin{description}
			\item[Objectives]\mbox{}\par
				\begin{itemize} 
					\item 
				\end{itemize} 
			\item[Setting] \mbox{}\par
				\begin{itemize} 
					\item US 6 cities
				\end{itemize} 
			\item[Outcome] \mbox{}\par
				\begin{itemize} 
					\item BMI continuous in 1-5 exams over 12 years
				\end{itemize} 
			\item[Exposure] \mbox{}\par
				\begin{itemize} 
					\item Spatial measure of neighborhood level racial/ethnic residential segregation
				\end{itemize} 
			\item[Confounders] \mbox{}\par
				\begin{itemize} 
					\item Time-invariant: age, sex, nativity, education, duration of residence, language. 
					\item Time-varying: working, marital status, income, smoking, cancer diagnosis, years since baseline.
				\end{itemize} 
			\item[Statistical analysis]\mbox{}\par
				\begin{itemize} 
					\item Stratified analysis by race. 
					\item Each specific segregation level was used for each race group.
				\end{itemize} 
			\item[Findings]\mbox{}\par
				\begin{itemize}
					\item 
				\end{itemize}
			\item[Strengths]\mbox{}\par
				\begin{itemize}
					\item
				\end{itemize}
			\item[Limitations]\mbox{}\par
				\begin{itemize}
					\item
				\end{itemize}
			\item[Significance]\mbox{}\par
				\begin{itemize} 
					\item Longitudinal analysis for racial/ethnic segregation and BMI. 
					\item Examining further if segregation impact BMI independent of neighborhood poverty. 
					\item This is important because neighborhood poverty is neighborhood deprivation - lack of healthy food outlets, physical activity facilities are often be a risk of obeisty, and understanding whether neighborhood poverty is a strong mediator? is critical for designing effective health policy interventions.
				\end{itemize} 
			\item[Conclusion]\mbox{}\par
				\begin{itemize} 
					\item Neighborhood segregation is associated with both higher and lower BMI by race and certain circumstances.
				\end{itemize} 
			\item[Takeaways] \mbox{}\par
				\begin{itemize}
					\item[$\clubsuit$] Ratial residential segregation is consistently associated with poor health condition and high mortality.
					\item[$\clubsuit$] Black segregation is associated with incresed mortality risk and adverse birth outcome, while Hispanic segregation is associated with stronger social cohesion, and protective against dicrimination and adoption of harmful behaviors.
					\item[$\clubsuit$] Higher segregation was negatively associated with BMI for white females and positively associated for Hispanic females in cross-secitonal analysis. In longitudinal analysis, higher segregation was positively associated with BMI for black females, but negatively associated for Hispanic females and males.
				\end{itemize} 
		\end{description}
%%%%%%%%%%%%%%%%%%%%%%%%%%%%%%%%%%%%%%%%%%%%

%%%%%%%%%%%%%%%%%%%%%%%%%%%%%%%%%%%%%%%%%%%%
\item{\bf Stafford(2007) -- 16-65+ years -- Cross-sectional study}
		\begin{description}
			\item[Objectives] \mbox{}\par
				\begin{itemize} 
					\item 
				\end{itemize} 
			\item[Settings and Subjects] \mbox{}\par
				\begin{itemize} 
					\item England and Scotland 
				\end{itemize} 
			\item[Outcome] \mbox{}\par
				\begin{itemize} 
					\item BMI, Waist hip ratio 
				\end{itemize} 
			\item[Exposure] \mbox{}\par
				\begin{itemize} 
					\item Crime, Policing, Physical dereliction, Highs street services, Leisure centres, Urban sprawl, Supermarkets, Fast-food outlets, Social disorder, Physical activity 
				\end{itemize} 
			\item[Confounders] \mbox{}\par
				\begin{itemize} 
					\item Gender, Age, Occupation 
				\end{itemize} 
			\item[Statistical analysis] \mbox{}\par
				\begin{itemize} 
					\item Structural Equasion model
				\end{itemize} 
			\item[Significance]\mbox{}\par 
				\begin{itemize} 
					\item First study to assess the causal path of modifiable neighborhood determinants to obesity.
				\end{itemize} 
			\item[Conclusion]\mbox{}\par 
				\begin{itemize} 
					\item Neighborhood disorders (e.g. crime, pilicing) and high street facilities were singnificantly associated with obesity. 
				\end{itemize} 
			\item[Takeaways] \mbox{}\par
				\begin{itemize}
					\item[$\clubsuit$] Neighborhood deprivation is related to morbidity, although relationship is weaker and smaller in magnitude than the relationships between individual SES and morbidity.
					\item[$\clubsuit$] It is important to unpack the "black box" on why, what kind of modifiable factors lie in the causal path. 
				\end{itemize} 
		\end{description}
%%%%%%%%%%%%%%%%%%%%%%%%%%%%%%%%%%%%%%%%%%%%

%%%%%%%%%%%%%%%%%%%%%%%%%%%%%%%%%%%%%%%%%%%%
\item{\bf Mckenbach(2014) -- NA -- Systematic review}
		\begin{description}
			\item[Setting] 
			\item[Outcome] 
			\item[Exposure] 
			\item[Confounders] 
			\item[Statistical analysis] 
			\item[Significance] Updated review of physical environment and obesity among adults focusing on high income countries. It distinguished the studies based on methodological quality, and made robust comparison.
			\item[Conclusion] Urban sprawl and land use mix were found to be associated with weight status only in North America. 
			\item[Takeaways] 
				\begin{itemize}\mbox{}\par
					\item[$\clubsuit$] Genetic factors underlie the propensity of individuals to become obese, but the rapid growth of obesity is hypothesized to be attributed from social and environmental causes. 
					\item[$\clubsuit$] There are heterogeneity of methods for measuring physical activity environments across continents. These should be addressed for future studies. 
				\end{itemize} 
		\end{description}
%%%%%%%%%%%%%%%%%%%%%%%%%%%%%%%%%%%%%%%%%%%%

%%%%%%%%%%%%%%%%%%%%%%%%%%%%%%%%%%%%%%%%%%%%
\item{\bf Sobal(1989) -- NA -- Systematic review}
		\begin{description}
			\item[Setting] 
			\item[Outcome] 
			\item[Exposure] 
			\item[Confounders] 
			\item[Statistical analysis] 
			\item[Significance] First revirew of all studies examined the association between SES and obesity
			\item[Conclusion] In developed societies, women with lower SES had higher obesity prevalence. In developing societies, men, women and children with higher SES had higher obesity prevalence.
			\item[Takeaways] 
				\begin{itemize}\mbox{}\par
					\item[$\clubsuit$] Enviroments are recognized as possible factors on obesity since Price 1987, and Stunkard 1986.
					\item[$\clubsuit$] The possiblity that SES might be related to body weight was first raised by Veblen 1889, in his speculation that thinness became an ideal of feminine beauty in chivalric times
					\item[$\clubsuit$] Energy balance is established mechanism of obesity Grarrow 1978.
					\item[$\clubsuit$] Energy intake - solely from food consumption. Energy output - basal metabolism, physical activity, thermic effect of food.
					\item[$\clubsuit$] Low prevalence of obesity among low SES group in devloping countries might be due to lack of food or even to famine, coupled with high energy expenditure 
					\item[$\clubsuit$] High prevalence of obesity among high SES group in developng countries might be due to affordability to obtain adequte food, or cultural preference to fat body shapes. Obesity might be a sign of health and wealth in devloping countries, which is opposit from developed countries.
					\item[$\clubsuit$] The higher propensty to become obesity can be often observed among people who are from countries experienced significant food shortage. Their body were biologically mutated to store fat to survive.   
					\item[$\clubsuit$] There is a huge difference between male and female how they percieve obesity as negative or stigmatized image in developed countries. 
				\end{itemize} 
		\end{description}

\end{itemize} %end of subsection
%%%%%%%%%%%%%%%%%%%%%%%%%%%%%%%%%%%%%%%%%%%%

%%%%%%%%%%%%%%%%%%%%%%%%%%%%%%%%%%%%%%%%%%%%
%%%%%%%%%%%%%%%%%%%%%%%%%%%%%%%%%%%%%%%%%%%%
\subsection{References about social environment}
\begin{itemize}
\item{\bf Coll-Planas(2016) -- >=60 years, 36 trials  -- Systematic review} 
		\begin{description}
			\item[Setting] NA 
			\item[Outcome] NA 
			\item[Exposure] NA
			\item[Confounders] NA
			\item[Statistical analysis] NA
			\item[Significance] First systematic review encompasses both mental and physical health of clinical interventions limited to randomized controll trial.
			\item[Conclusion] Social capital might promote general health of population, butmore high-quality research are needed.
			\item[Takeaways] \mbox{}\par
				\begin{itemize}
					\item[$\clubsuit$] Social capital is an umbrella concept - Structual (objective aspect): social networks, social contacts and participation belonging. Cognitive (subjective): social support, sense of belonging and trust. Itcan be further devided into - Bonding (ties in homogeenous group), Bridging (ties between hetrogeneous group), Linking (ties between people with unequal wealth, power and status).
				\end{itemize} 
		\end{description}
%%%%%%%%%%%%%%%%%%%%%%%%%%%%%%%%%%%%%%%%%%%%

%%%%%%%%%%%%%%%%%%%%%%%%%%%%%%%%%%%%%%%%%%%%
\item{\bf Ehsan(2015) -- 39 studies  -- Systematic review} 
		\begin{description}
			\item[Setting] NA 
			\item[Outcome] NA 
			\item[Exposure] NA
			\item[Confounders] NA
			\item[Statistical analysis] NA
			\item[Significance] After 2005, more research have been done both cross-sectionally and longitudinally on the association between SC and mental health. This study updates the review of recent findings. 
			\item[Conclusion] Common mental disorders were associeted with individual cognitive SC. 
			\item[Takeaways] \mbox{}\par
				\begin{itemize}
					\item[$\clubsuit$] The association between social determinants, such as low-ses, unemploynment, low educational attainment and negative health concequences are well established. - Social capital can extend the concept of social deteminants by including quantity and quality of social relationships.
					\item[$\clubsuit$] The association between "cognitive" SC and mental health was found in te systematic review by De Silva et al. 2005. 
				\end{itemize} 
		\end{description}
%%%%%%%%%%%%%%%%%%%%%%%%%%%%%%%%%%%%%%%%%%%%

%%%%%%%%%%%%%%%%%%%%%%%%%%%%%%%%%%%%%%%%%%%%
\item{\bf McNeill(2006) --   -- Systematic review} 
		\begin{description}
			\item[Setting] NA 
			\item[Outcome] NA 
			\item[Exposure] NA
			\item[Confounders] NA
			\item[Statistical analysis] NA
			\item[Significance] Conceptualizing social envioronment by defining each component related to physical activity. Inconsistent terminology is often used for defining social environment, so conceptualizing is critical to facilitate further high quality studies.
			\item[Conclusion] 
			\item[Takeaways] \mbox{}\par
				\begin{itemize}
					\item[$\clubsuit$] Environmental and individual factors shape simultaneously behavior based on social-cognitive theory.
					\item[$\clubsuit$]
				\end{itemize} 
		\end{description}

\end{itemize} %end of subsection

%%%%%%%%%%%%%%%%%%%%%%%%%%%%%%%%%%%%%%%%%%%%
%%%%%%%%%%%%%%%%%%%%%%%%%%%%%%%%%%%%%%%%%%%%
\subsection{Neighborhood environment and longitudinal change of BMI}
\begin{itemize}
	\item{\bf Rachele(2017) -- 40-65 years -- longitudinal study} 
		\begin{description}
			\item[Objectives] \mbox{}\par
				\begin{itemize}
					\item 
				\end{itemize} 
			\item[Settings and Subjects]\mbox{}\par 
				\begin{itemize}
					\item Brisbane, Australia
				\end{itemize} 
			\item[Outcome] \mbox{}\par
				\begin{itemize}
					\item BMI (self-reported)
				\end{itemize} 
			\item[Exposure] \mbox{}\par
				\begin{itemize}
					\item Index of Relative Socioeconomic Disadvantage for each census tract.
				\end{itemize} 
			\item[Confounders] \mbox{}\par
				\begin{itemize}
					\item age, education, occupation, income, neighborhood self-selection.
				\end{itemize} 
			\item[Statistical analysis] \mbox{}\par
				\begin{itemize}
					\item 3-level mixed-effects linear regression models. 
					\item Observations varied overtime within individuals = level1, observations clustered within individuals who were specified = level2, individuals clustered within neighborhood = level3. 
					\item Participants who returned to the study after a nonresponse and had moed were included. 
					\item All variables were  observed at each wave (0, 1, 2, 3).
				\end{itemize}
			\item[Significance] \mbox{}\par
				\begin{itemize}
					\item Longitudinal model taking into account the change of neighborhood enviornment.
				\end{itemize}
			\item[Conclusion] \mbox{}\par
				\begin{itemize}
					\item It may not be causal for the association between neighborhood socioeconomic disadvantage and BMI among middle aged and older adults.
				\end{itemize}
			\item[Takeaways] \mbox{}\par
				\begin{itemize}
					\item[$\clubsuit$] The between-individual coefficients show that women living in more disadvantaged neighborhoods had, on average, a higher BMI, but not for men. 
					\item[$\clubsuit$] The within-individual coefficients show that moving to a new neighborhood with a 1-quintile increase in disadvantage was not associated with a within-individual increase in BMI.
				\end{itemize}
		\end{description}
\end{itemize}
%%%%%%%%%%%%%%%%%%%%%%%%%%%%%%%%%%%%%%%%%%%%

%%%%%%%%%%%%%%%%%%%%%%%%%%%%%%%%%%%%%%%%%%%%
% Swedish Annual Level of Living Survey
%%%%%%%%%%%%%%%%%%%%%%%%%%%%%%%%%%%%%%%%%%%%
\section{SALLS}
\subsection{Neighborhood environment and cardiovascular diseases}
\begin{itemize}
	\item{\bf Sundquist(1999) -- 25-74 years -- Cross-sectional study -- Int J Epidemiol} 
		\begin{description}
			\item[Objectives]\mbox{}\par
				\begin{itemize}
					\item To examine whether CVD risk factors, such as BMI, physical activiy and smoking diffred among individuals living in different neighborhoods when adjusted for age and individual SES.
				\end{itemize}
			\item[Setting] \mbox{}\par
				\begin{itemize}
					\item  Sweden 
				\end{itemize}
			\item[Outcome]\mbox{}\par
				\begin{itemize}
					\item BMI, physical activity, smoking
				\end{itemize}
			\item[Exposure] \mbox{}\par
				\begin{itemize}
					\item The social status defined in terms of the Care Need Index and Townsend score.
				\end{itemize}
			\item[Confounders]\mbox{}\par 
				\begin{itemize}
					\item Basic, education
				\end{itemize}
			\item[Statistical analysis] \mbox{}\par
				\begin{itemize}
					\item Hierarchial logistic regresion was used. The individual factors were treated as fixed effects. The CNI and interactions with the CNI were treated as random effects.
				\end{itemize}
			\item[Significance] \mbox{}\par
				\begin{itemize}
					\item First study investigated the effect of neighborhood environment on risk factors of CVD with nation-wide random samples after controlling for individual SES. 
				\end{itemize}
			\item[Conclusion] \mbox{}\par
				\begin{itemize}
					\item After adjusting for individual age, sex, and SES, people living in deprived neighborhood had a higher risk of obesity, smoking, and physical inactivity.
				\end{itemize}
			\item[Takeaways] \mbox{}\par
				\begin{itemize}
					\item[$\clubsuit$] The SALLS is based on face-to-face interviews that generally took place in the respondents' homes.
					\item[$\clubsuit$] SALLS matched with social status of the respondents' residental areas measured as the CNI and the Townsend score was used.
					\item[$\clubsuit$] Excluded 837 SMAS area with <50 inhabitants
				\end{itemize}
		\end{description}
%%%%%%%%%%%%%%%%%%%%%%%%%%%%%%%%%%%%%%%%%%%%%%%%%%%%%

%%%%%%%%%%%%%%%%%%%%%%%%%%%%%%%%%%%%%%%%%%%%%%%%%%%%%
	\item{\bf Cubbin(2006) -- 25-74 years --  Cross-sectional -- Scand J Pub Health} 
		\begin{description}
			\item[Objectives]\mbox{}\par
				\begin{itemize}
					\item To examine whether neighborhood deprivation is independently associated with CVD risk factors, such as smoking, physical inactiviy, obesity, diabetes, and hypertension for Swedish population. 
				\end{itemize}
			\item[Setting] \mbox{}\par
				\begin{itemize}
					\item Sweden 
				\end{itemize}
			\item[Outcome]\mbox{}\par
				\begin{itemize}
					\item Smoking, physical inactivity, obesity, diabetes, hypertension (self reported)
				\end{itemize}
			\item[Exposure] \mbox{}\par
				\begin{itemize}
					\item SES composite index by education, occupation, parents' occupation. 
					\item Deprivation index by elderly living alone, foreign born, residents who moved, unemployed, single parents, low education status, chidlren under age 5.
				\end{itemize}
			\item[Confounders] \mbox{}\par
				\begin{itemize}
					\item Age, gender, marital status, immigration status, urbanization, and a composite measure of SES
				\end{itemize}
			\item[Statistical analysis] \mbox{}\par
				\begin{itemize}
					\item Hierarchial logistic regresion was used. The individual factors were treated as fixed effects. The CNI and interactions with the CNI were treated as random effects.
				\end{itemize}
			\item[Significance] \mbox{}\par
				\begin{itemize}
					\item Builded the previous study by using more recent data (1996-2000), additonal CVD risk factors, and a more comprehensive measure of individual SES 
				\end{itemize}
			\item[Conclusion] \mbox{}\par
				\begin{itemize}
					\item After adjusting for individual age, sex, and SES, people living in deprived neighborhood had a higher risk of obesity, smoking, and physical inactivity.
				\end{itemize}
			\item[Takeaways] \mbox{}\par
				\begin{itemize}
					\item[$\clubsuit$] The associations between neighborhood deprivation and CVD has been found by numerous studies. Afterwards, the associations between neighborhood deprivation and CVD risk factors such as smoking, physical inactivity, obesiy, etc has been found after controlling for a range of individual SES.
					\item[$\clubsuit$] Previous research has shown that labor immigrants from Finland, the largenst immigrant group in Sweden most of whome immigrated for work have poorer health than other labor immigrants who have settled in Sweden.
				\end{itemize}
		\end{description}
\end{itemize} %end of subsection
%%%%%%%%%%%%%%%%%%%%%%%%%%%%%%%%%%%%%%%%%%%%%%%%%%%%%


%%%%%%%%%%%%%%%%%%%%%%%%%%%%%%%%%%%%%%%%%%%%%%%%%%%%%
%%%%%%%%%%%%%%%%%%%%%%%%%%%%%%%%%%%%%%%%%%%%%%%%%%%%%
\subsection{Neighborhood environment and mental health}
\begin{itemize}
	\item{\bf Lofors(2006) -- 25-64 years -- Cross-sectional study -- Euro J Pub Health} 
		\begin{description}
			\item[Objectives]\mbox{}\par
				\begin{itemize}
					\item To examine an association between neighborhood income and anxiety after adjusting for individual demographic, socio-economic and social characteristics.
				\end{itemize}
			\item[Setting and subjects] \mbox{}\par
				\begin{itemize}
					\item Sweden. National random sample of the entire Swedish population: 15659 women and 15225 men.
				\end{itemize}
			\item[Outcome]\mbox{}\par
				\begin{itemize}
					\item  Self-reported anxiety. "Do you suffer from nervousness, uneasiness, or anxiety? 1. Yes, severe problems, 2. Yes, slight problems, 3. No."
				\end{itemize}
			\item[Exposure] \mbox{}\par
				\begin{itemize}
					\item Proportions of people with the lowest national income quartile at SAMS, and it was divided into quartiles (Q1=least proportion of people with low income, Q4=highest proportion of people with low income). 
					\item The income variable was based on annual disposable family income and the number of people in the family was adjusted. 
					\item Also weighted small chidren as they give lower consumption. 
					\item Source data was from December 1997 women and men aged 25-64. 
					\item Total of 7094 SAMS were included for this study population. 
					\item The home addresses were geocoded previously. 
					\item 239 subjects who were not linked to SAMS were categorized in the neighborhood with the highest income in order to avoid an overestimation of RR.
				\end{itemize}
			\item[Confounders] \mbox{}\par
				\begin{itemize}
					\item Age (10y category), gender, marital status, immigrant status, social networks (1. having at least one close friend, 2. meeting friends and other acquantances, 3. exchanging favours with neighbors, 4. casual interaction with neighbors, 5. parents alive), housing tenure, employment, income.
				\end{itemize}
			\item[Statistical analysis] \mbox{}\par
				\begin{itemize}
					\item A log binomial model to estimatee prevalence ratios. 
					\item Stepwise inclusion of individual variables with 4 models.
				\end{itemize}
			\item[Findings] \mbox{}\par
				\begin{itemize}
					\item There was an significant gradient association between neighborhood income and prevalence of anxiety when only adjusting for age and gender. 
					\item However, the association dissapeared when other confounders were included.
				\end{itemize}
			\item[Strengths] \mbox{}\par
				\begin{itemize}
					\item 8 individual potential confounders collected by well-trained interviewers, which means that they are adequately measured and lower the risk of residual confoundings. 
					\item Large random sample. Neighborhood income was based on entire Swedish population but not the study population.
				\end{itemize}
			\item[Limitations] \mbox{}\par
				\begin{itemize}
					\item Non respondents (20\%) might have caused response bias. 
					\item Anxiety is self-reported. 
					\item Multilevel modeling was not possible as the number of individuals per SAMS was too samll to allow a correct calculation of the variance measure.
				\end{itemize}
			\item[Significance] \mbox{}\par
				\begin{itemize}
					\item The study examined the contextual effect of neighborhood income on anxiety, which is a most common mental disorders among working age group by using national level random samples with control of individual (compositional) effects.
				\end{itemize}
			\item[Conclusion] \mbox{}\par
				\begin{itemize}
					\item The contextual effect of neighborhood income disappeared after controling for individual (compositional) effects.
				\end{itemize}
			\item[Takeaways] \mbox{}\par
				\begin{itemize}
					\item[$\clubsuit$] Anxiety tends to be chronic, and can be as disabling as somatic disorders. In addition, individuals with anxiety place a high strain upon the whole healthcare system.
					\item[$\clubsuit$] SALLS has been conducted by Statistics Sweden since 1975 and consists of interviews performed at home by trained interviewers.
					\item[$\clubsuit$] SALLS consists of 11 areas of investigation including demographic and socio-economic characteristics, health status and social environment.
				\end{itemize}
		\end{description}
\end{itemize} %end of subsection

%%%%%%%%%%%%%%%%%%%%%%%%%%%%%%%%%%%%%%%%%%%%%%%%%%%%%
%%%%%%%%%%%%%%%%%%%%%%%%%%%%%%%%%%%%%%%%%%%%%%%%%%%%%
\subsection{Longitudinal trend of exercise}
\begin{itemize}
	\item{\bf Leijon(2015) -- 16-63 years -- Longitudinal study -- Pop Health Met} 
		\begin{description}
			\item[Objectives]\mbox{}\par
				To examine the longitudinal effect of age group and birth cohort on regular exercise.
			\item[Setting and subjects]\mbox{}\par 
				\begin{itemize}
					\item Sweden. 
					\item Data from 4 time points: 1980-1981, 1988-1989, 1996-1997, 2004-2005. 
					\item Including all individuals who had answered at least once, completed with new individuals in the age span 16-23 years. 
					\item Individuals 16-63 years were included.
					\item The non-response rate varies between 20-25\% by time frame.
					\item The response patterns for all 4 surveys were similar with regard to sex, age, place of living and salary.
				\end{itemize}
			\item[Outcome]\mbox{}\par
				\begin{itemize}
					\item Self-report excise. 0. never or now and then vs. 1. regularly more than once a week. 
					\item Following original question was dichotomized. How much do you exercise in your leisure time? 1. I get practically no exercise atall, 2. I exercise occasionally, 3. I exercise regularly about once a week, 4. I exercise regularly twice a week, 5. I exercise regularly and vigorously at least twice a week.
				\end{itemize}
			\item[Exposure] \mbox{}\par
				\begin{itemize}
					\item Age at the time of the interview. 
					\item Birth cohort (based on year of birth)
				\end{itemize}
			\item[Confounders] \mbox{}\par
				\begin{itemize}
					\item Gender, education level, BMI, smoking, self-reported health status.
				\end{itemize}
			\item[Statistical analysis]\mbox{}\par 
				\begin{itemize}
					\item Distribution of the explanatory variables.
					\item Prevalence of exercise in the different confounders.
					\item GLMM with random intercepts and random slopes to test the change in regular exercise over time according to age group and cohort.
					\item Model I: age, cohort and the interaction age-by-cohort, age-squared, and cohort-squared.
					\item Model II: adjusted for all confounders.
				\end{itemize}
			\item[Findings] \mbox{}\par
				\begin{itemize}
					\item The prevalence of regular exercise increased over time in all studied sub-groups among both men and women.
					\item Differences related to educational level increased over time.
					\item For women, all birth cohorts increased the prevalence of exercise.
					\item For men, only the 3 oldest birth cohorts increased the prevalence of exercise, and the 3 youngest birth cohorts decreased.
				\end{itemize}
			\item[Strengths] \mbox{}\par
				\begin{itemize}
					\item Long period follow-up with repeased measurements.
					\item SALLS is a simple random sample with a longitudinal panel with repeate measurements, which can represent the entire Swedish population.
					\item Period effect can be excluded and substituted by the interaction between age nad birth cohort.
				\end{itemize}
			\item[Limitations] \mbox{}\par
				\begin{itemize}
					\item Outcome is self-report. Public awareness of the importance of physical activity may have increased during the study period.
					\item Non-response might have overestimated the activity level as those inactive could havee bbeen overrepresented among non-responders.
				\end{itemize}
			\item[Significance]\mbox{}\par
				\begin{itemize}
					\item Identified an increase trend of physical activity in the population level.
					\item Identified a difference of the increase trend of PA by educational level.
					\item Identified a decrease trend of PA among middle age group. 
				\end{itemize}
			\item[Conclusion]\mbox{}\par
				\begin{itemize}
					\item The prevalence of PA has been increasing over time in Sweden. However, the difference inthe prevalence of PA by educational level has been increasing, and there is a negative trend among a certain younger birth cohort group. It is important to implement policy to reduce this potential inequity and target early age group to reduce the risk for ill health.
				\end{itemize}
			\item[Takeaways] \mbox{}\par
				\begin{itemize}
					\item[$\clubsuit$] Monitoring trends in populations is important for making it possible to identify and target certain groups in the population when planning interventions.
				\end{itemize}
		\end{description}
\end{itemize} %end of subsection

%%%%%%%%%%%%%%%%%%%%%%%%%%%%%%%%%%%%%%%%%%%%%%%%%%%%%
%%%%%%%%%%%%%%%%%%%%%%%%%%%%%%%%%%%%%%%%%%%%%%%%%%%%%
\subsection{Longitudinal trend of mental health}
\begin{itemize}
	\item{\bf Calling(2017) -- 16-71 years -- Longitudinal study -- BMC Psychiatry} 
		\begin{description}
			\item[Objectives]\mbox{}\par
				To examine the longitudinal change of prevalence of self-reported anxiety by different birth cohorts and age groups.
			\item[Setting and subjects]\mbox{}\par 
				\begin{itemize}
					\item Sweden. 
					\item Data from 4 time points: 1980-1981, 1988-1989, 1996-1997, 2004-2005.
					\item Including all individuals who had answered at least once, completed with new individuals in the age span 16-23 years. 
					\item Individuals 16-71 years were included.
					\item Excluded individuals with missing values for weight or height (1\%), or PA.
					\item Individuals with missing values for educational level was classified as the highest educational level.
				\end{itemize}
			\item[Outcome]\mbox{}\par
				\begin{itemize}
					\item Self-report anxiety. 0. no, 1. yes mild or severe.
					\item Following original question was dichotomized. "Do you suffer from nervousness, uneasiness, r anxiety?". 0. no, 1. yes mild, 2. yes severe.
				\end{itemize}
			\item[Exposure] \mbox{}\par
				\begin{itemize}
					\item 3 time-related variables.
					\item Assessment period.
					\item Age at the time of the interview. 
					\item Birth cohort (based on year of birth).
				\end{itemize}
			\item[Confounders] \mbox{}\par
				\begin{itemize}
					\item Gender, education level, urbanization, marital status, smoking, leisure time PA, BMI.
					\item Included in the models as time-varying variables.
				\end{itemize}
			\item[Statistical analysis]\mbox{}\par 
				\begin{itemize}
					\item Distribution of the explanatory variables.
					\item Prevalence of anxiety in the different confounders.
					\item GLMM with random intercepts and random slopes to test the change in prevalence of anxiety over time according to age groups and birth cohorts.
					\item The effect of time period does not need to be estimated for a longitudinal panel study, as age and time express the same effect.
					\item Tested interactions with time and each confoundings, but none were significant.
					\item Model I: age, cohort and the interaction age-by-cohort.
					\item Model II: adjusted for all confounders.
				\end{itemize}
			\item[Findings] \mbox{}\par
				\begin{itemize}
					\item Overall prevalnce of self-reported anxiety increased except for the oldest age groups 64-71 years.
					\item The highest increase was found in young adults 16-23 years.
				\end{itemize}
			\item[Strengths] \mbox{}\par
				\begin{itemize}
					\item Long period follow-up with repeased measurements.
					\item SALLS is a simple random sample with a longitudinal panel with repeate measurements, which can represent the entire Swedish population.
					\item Period effect can be excluded and substituted by the interaction between age nad birth cohort.
					\item Longitudina mixed model made it possible to distinguish changes over time within individuals (age effects) from differences among individuals at baseline (cohort effects).
				\end{itemize}
			\item[Limitations] \mbox{}\par
				\begin{itemize}
					\item Outcome is self-report. Some have been misclassified or underreported. 
					\item Non response might have underestimated the prevalence of anxiety as those with anxiety might have not responded.
					\item Several other risidual confoundings such as immigration status, unemploynment, alcohol and drugs.
				\end{itemize}
			\item[Significance]\mbox{}\par
				\begin{itemize}
					\item Identified longitudinal trend of the prevalence of anxiety by controlling by age or birth cohort effects.
					\item 
				\end{itemize}
			\item[Conclusion]\mbox{}\par
				\begin{itemize}
					\item The prevalence of anxiety has been increasing over time except older age group. Studies to disentangle the reasons of increased anxiety are needed.
				\end{itemize}
			\item[Takeaways] \mbox{}\par
				\begin{itemize}
					\item[$\clubsuit$] Recent US and Canadian surveys revealed that people born in the oldest and more recently born birth cohorts had higher levels of psychological distress than those cohorts born in mid-century.
					\item[$\clubsuit$] Since 1979, there have been four main themes in the SALLS: social relations, work, health, physical environment. 
					\item[$\clubsuit$] Due to increased educational demants at the labor markets, there are many young adults being unemployed.
				\end{itemize}
		\end{description}
\end{itemize} %end of subsection

%%%%%%%%%%%%%%%%%%%%%%%%%%%%%%%%%%%%%%%%%%%%%%%%%%%%%
%%%%%%%%%%%%%%%%%%%%%%%%%%%%%%%%%%%%%%%%%%%%%%%%%%%%%
\subsection{Change in lifestyle and health and mortality}
\begin{itemize}
	\item{\bf Johansson(1999) -- 25-74 years -- Cross-sectional and longitudinal study -- Int J Epidemiol} 
		\begin{description}
			\item[Objectives]\mbox{}\par
				\begin{itemize}
					\item To examine the change in lifestye and its influence on health and all cause mortality.
				\end{itemize}
			\item[Setting and subjects]\mbox{}\par 
				\begin{itemize}
					\item Sweden. 
					\item Data from 2 time points: 1980-1981, 1988-1989.
					\item 3843 adults 25-74 years from the same random sample.
				\end{itemize}
			\item[Outcome]\mbox{}\par
				\begin{itemize}
					\item Self-report health status. 0. good vs. 1. poor or anywhere between good and poor.
					\item Original question was dichotomized. How would you describe your general health? 1. good. 2. poor. 3. poor or anywhere between good and poor.
					\item All-cause mortality. Person-year at risk were calculated from the date of the second interview until death or, for those who survived, until 31 Dec 1995.
				\end{itemize}
			\item[Exposure] \mbox{}\par
				\begin{itemize}
					\item Physical activity. 0. regular physical  activity at least once a week. 1. being physically inactive or occasiionally active. 
					\item Smoking. 0. Never smoked. 1. Former smoker. 2. Daily smoker, 1-14 g/day. 3. Daily smoker, $\ge$14 g/day.
					\item BMI
				\end{itemize}
			\item[Confounders] \mbox{}\par
				\begin{itemize}
					\item Age, marital status, country of birth, education level, self-reported hypertension.
				\end{itemize}
			\item[Statistical analysis]\mbox{}\par 
				\begin{itemize}
					\item Age-standardized prevalences of poor health status by each variable by year.
					\item GEE for poor health vs all variables cross-sectionally.
					\item For GEE, the regression and within-subject correlations are modelled separetely.
					\item Unconditional logistic regression for the transitional model: outcome is self-reported health status in the 2nd time point.
					\item Values from both time points: physical activity, BMI and smoking were included.
					\item A transition model takes into account both the regression objective and the within-subject correlation.
					\item Proportional hazard models was applied for all-cause mortality.
					\item Changes in lifestyle factors, i.e. physical activity, smoking, BMI, hypertension were defined as 4 categories. 1. remained high risk. 2. changed from low to high risk. 3. changed from high to low risk. 4. remained low risk.
					\item Education, country of birth, and marital status were adjusted.
				\end{itemize}
			\item[Findings] \mbox{}\par
				\begin{itemize}
					\item For the cross-sectional association between lifestyle and health status, physical activity and BMI were independently associated with health. 
					\item Smoking was not independently associated, but significant interaction was found with physical activity based on the cross-secitonal GEE.
					\item Physical activity was found among the strongest factor for health status based on the cross-sectional GEE.
					\item From transitional model, those who stopped PA or remained no PA had about 2 times higher odds for poor health.
					\item From hazard model, those who stopped PA or remained no PA had higher risk of death.
				\end{itemize}
			\item[Strengths] \mbox{}\par
				\begin{itemize}
					\item Nationally representative sample.
					\item Marginal model and transitional model was effective to analyze the change.
				\end{itemize}
			\item[Limitations] \mbox{}\par
				\begin{itemize}
					\item Only 2 time points, which means changes might have happend during 8 years.
					\item Health status and others were based on self-reported.
				\end{itemize}
			\item[Significance]\mbox{}\par
				\begin{itemize}
					\item Identified a cross-sectional associations between smoking, obesity, PA and health status.
					\item Identified the change of lifestyle behaviors were associated with all-cause mortality.
				\end{itemize}
			\item[Conclusion]\mbox{}\par
				\begin{itemize}
					\item Among smoking, PA, and BMI, PA was found to be an important factor to maintain good health and achieving low mortality.
				\end{itemize}
			\item[Takeaways] \mbox{}\par
				\begin{itemize}
					\item[$\clubsuit$] During the 1980s, mortality declined in Sweden as in most Western countries.
					\item[$\clubsuit$] Most parts of the mortality decline was attributed to the decrrese in the rates of CVD and stroke.
				\end{itemize}
		\end{description}
\end{itemize} %end of subsection

%%%%%%%%%%%%%%%%%%%%%%%%%%%%%%%%%%%%%%%%%%%%%%%%%%%%%
%%%%%%%%%%%%%%%%%%%%%%%%%%%%%%%%%%%%%%%%%%%%%%%%%%%%%
\subsection{Migration and health}
\begin{itemize}
	\item{\bf Westman (2008) -- unknown -- Cross-sectional -- Scand J Pub Health} 
		\begin{description}
			\item[Objectives]\mbox{}\par
				\begin{itemize}
					\item To examine the prevalence of poor health between Finns living in Sweden and Finns living in Finland.
				\end{itemize}
			\item[Setting and subjects]\mbox{}\par 
				\begin{itemize}
					\item Finland and Sweden. 
					\item SALLS between 1996-2003.
					\item Finnish national survey "Health 2000".
				\end{itemize}
			\item[Outcome]\mbox{}\par
				\begin{itemize}
					\item Self-report health status. 0. good vs. 1. poor or anywhere between good and poor.
					\item Original question was dichotomized. 1. very good. 2. good. 3. fair. 4. poor. 5. very poor.
				\end{itemize}
			\item[Exposure] \mbox{}\par
				\begin{itemize}
					\item Country of birth and residence: 1. Finns living in Sweden, 2. Finns living in Finland, 3. Swedes living in Sweden.
				\end{itemize}
			\item[Confounders] \mbox{}\par
				\begin{itemize}
					\item Age, marital status, education level, employment status, smoking.
				\end{itemize}
			\item[Statistical analysis]\mbox{}\par 
				\begin{itemize}
					\item Multiple variables logistic regression considering the strata and primary sampling units of Finnish Health 2000.
				\end{itemize}
			\item[Findings] \mbox{}\par
				\begin{itemize}
					\item Odds of poor health was significantly higher among Finns regardless to where they live compared to Swedes.
					\item Finnis women living in Sweden had higher odds of poor health compared to those living in Finland.
					\item Finnis men living in Sweden had lower odds of poor health compared to those living in Finland.
				\end{itemize}
			\item[Strengths] \mbox{}\par
				\begin{itemize}
					\item Nationally representative sample.
				\end{itemize}
			\item[Limitations] \mbox{}\par
				\begin{itemize}
					\item Cross-sectional associations.
				\end{itemize}
			\item[Significance]\mbox{}\par
				\begin{itemize}
					\item Identified the difference of prevalence in poor health between Swedes and Finns, as well as Finns immigrants in Sweden and native Finns.
				\end{itemize}
			\item[Conclusion]\mbox{}\par
				\begin{itemize}
					\item Migration might have different effect on health by gender.
				\end{itemize}
			\item[Takeaways] \mbox{}\par
				\begin{itemize}
					\item[$\clubsuit$] Finnish migrants is the largest immigrant group - 20\% of immigrant population in Sweden.
				\end{itemize}
		\end{description}
\end{itemize} %end of subsection

%%%%%%%%%%%%%%%%%%%%%%%%%%%%%%%%%%%%%%%%%%%%%%%%%%%%%
%%%%%%%%%%%%%%%%%%%%%%%%%%%%%%%%%%%%%%%%%%%%%%%%%%%%%
\subsection{Psychosocial working conditions and health}
\begin{itemize}
	\item{\bf Sundquist (2003) -- 25-64 years -- Cross-sectional -- Ethnicity and Health} 
		\begin{description}
			\item[Objectives]\mbox{}\par
				\begin{itemize}
					\item To examine the associations between high psychological job depands and low decision latitude, work related social support and long term illness among immigrants and Swedish-born people.
				\end{itemize}
			\item[Setting and subjects]\mbox{}\par 
				\begin{itemize}
					\item Sweden. 
					\item SALLS between 1994-1997.
					\item 25-64 years.
					\item 10,072 Swedish-born persons, 710 labour migrants, 333 refugees.
				\end{itemize}
			\item[Outcome]\mbox{}\par
				\begin{itemize}
					\item Self-report long-term illness. 0. no, vs. 1. yes.
					\item Original question was dichotomized. 
				\end{itemize}
			\item[Exposure] \mbox{}\par
				\begin{itemize}
					\item Psychological job demands. 0. low, 1. medium, 2. high.
					\item Job decision latitude. 0. sometimes, 1. often, 2. flexible working hours.
					\item Job strain: psycological job demands + job decision latitude.
					\item Work-related social support index: can interrupt work to talk with co-workers; meets co-workers outside the workplace.
				\end{itemize}
			\item[Confounders] \mbox{}\par
				\begin{itemize}
					\item Age, marital status, migration status, socio-economic status.
				\end{itemize}
			\item[Statistical analysis]\mbox{}\par 
				\begin{itemize}
					\item Unconditional logistic regression.
					\item Hosmer-Lemshow test and residual analysis was used to assess model fits.
					\item Interaction tests were conducted.
				\end{itemize}
			\item[Findings] \mbox{}\par
				\begin{itemize}
					\item Refugees had a higher risk of long-term illness than Swedes.
					\item Refugees with low work-related social support had a high risk of long-term illness.
				\end{itemize}
			\item[Strengths] \mbox{}\par
				\begin{itemize}
					\item Nationally representative sample.
				\end{itemize}
			\item[Limitations] \mbox{}\par
				\begin{itemize}
					\item Cross-sectional associations.
				\end{itemize}
			\item[Significance]\mbox{}\par
				\begin{itemize}
					\item Identified the risk of long-term illness among refugees with low work-related social support and conditios.
				\end{itemize}
			\item[Conclusion]\mbox{}\par
				\begin{itemize}
					\item Psychosocial working conditions are important especially among foreign-born people for their health.
				\end{itemize}
			\item[Takeaways] \mbox{}\par
				\begin{itemize}
					\item[$\clubsuit$] 
				\end{itemize}
		\end{description}
\end{itemize} %end of subsection




%%%%%%%%%%%%%%%%%%%%%%%%%%%%%%%%%%%%%%%%%%%%
% Studies for older adults
%%%%%%%%%%%%%%%%%%%%%%%%%%%%%%%%%%%%%%%%%%%%
\section{Studies for older adults}
\subsection{BMI and mortality}
\begin{itemize}	
	\item{\bf Winter(2014) -- $\geq$65 years -- Meta-analysis -- Am J Clin Nutr} 
		\begin{description}
			\item[Objectives]\mbox{}\par
				\begin{itemize}
					\item To examine the association betweeen BMI and all-cause mortality among older adults.
				\end{itemize}
			\item[Search database] \mbox{}\par
				\begin{itemize}
					\item MEDLINE, CINAHL, Cochrane Library 1990-2013 for research articles. 
					\item MEDLINE 2010-2013 for review articles. 
					\item Studies from Europe, North America, and Australia.
				\end{itemize}
			\item[Search terms]\mbox{}\par
				\begin{itemize}
					\item body mass index OR BMI OR weight AND mortality AND old* OR geriatr* OR senior for research articles. 
					\item body mass index OR obesity ND mortality NOT institute* OR hospital* OR nursing home for review articles.
				\end{itemize}
			\item[Inclusion and Exclusion] \mbox{}\par
				\begin{itemize}
					\item Inclusion: prospective cohort AND $\geq$65 years AND reported RRs or HRs AND minimum follow-up period $\geq$5 years, baseline BMI and smoking status.
					\item Exclusion: HRs reported only for weight in kg OR reported $\le$3 categories of BMI OR wholly nonwhite populations.
				\end{itemize}
			\item[Statistical analysis] \mbox{}\par
				\begin{itemize}
					\item A 2-stage random-effects meta-analysis for a potential nonlinear relation between BMI and all-case mortality.
				\end{itemize}
			\item[Significance] \mbox{}\par
				\begin{itemize}
					\item While most studies focus on overweight or obese among older adults, this study identified an increased risk of mortality among those with BMI $\le$23.0 which is in the range of healthy weight status by the WHO definition. 
					\item In addition, this findings were confirmed by subgroup analysis by smoking status, which implies that lower BMI was independently associated with mortality regardless preexisting potential risk factors.
				\end{itemize}
			\item[Conclusion] \mbox{}\par
				\begin{itemize}
					\item Signifcant increase of mortality risk was observed from BMI $\le$23.0. 
				\end{itemize}
			\item[Takeaways] \mbox{}\par
				\begin{itemize}
					\item[$\clubsuit$] The WHO defines a healthy body weight range for adddults as a BMI between 18.5~24.9, but those are based on younger adults.
					\item[$\clubsuit$] Weight change ay be more important for older adults in terms of health risks.
				\end{itemize}
		\end{description}

\end{itemize}


%%%%%%%%%%%%%%%%%%%%%%%%%%%%%%%%%%%%%%%%%%%%
% Health outocomes in general
%%%%%%%%%%%%%%%%%%%%%%%%%%%%%%%%%%%%%%%%%%%%
\section{Burdens of various health outcomes}
\subsection{Physical inactivity}
\begin{itemize}
	\item Physical inactivity is the fourth-leading cause of chronic disease and mortality such as heart disease, stroke, diabetes, and cancers, contributing to over 3 million preventable deaths annually worldwide (WHO. Global recommendations on physical activity for health. Geneva. WHO; 2010. p. 58.)
	\item 60\% of adult population in the world does not reach the recommended level of physical activity (WHO. The World health report:2002:Reducing the riss, promoting healthy life. Geneva. WHO; 2002. p.248.)
	\item Physical inactivity is the fifth leading risk facto for burden of disease for women in Sweden and the sixth leading risk factor for men (Agardh E, Moradi T, Allebeck P. The contribution of risk factors to the burden of disease in Sweden. A comparison between Swedish and wHO data. Lakartidningen. 2008.)
	\item Exercise is defined as "planned, structured, and repetitive and purposive in the sense that the improvement or maintenance of one or more components of physical fitness is the objective", subcategory of physical activity (Physical Activity Guidelines Advisory Committee. Physical Activity Guidelines Advisory Committee Report, 2008.)
\end{itemize}

\subsection{Mental illness}
\begin{itemize}
	\item Mental illness including anxiety is oone of the leading causes of disability worldwide (WHO. The World Health report 2001. Mental health, new understanding, new hope. Geneva: 2001; Murracy CJ, Lopez AD. Global mortality, disability, and the contribution of risk factors: Global Burden of Disease Study. Lancet. 1997;349:1436-42.)
	\item 
	\item 
	\item 
\end{itemize}





\end{document}
