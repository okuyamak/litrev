\documentclass{article}
%\usepackage{pdflscape}
%\usepackage{afterpage}
%\usepackage{capt-of}
\usepackage{booktabs}
%\usepackage{tabulary}
%\usepackage{rotating}	
%\newcommand{\tabitem}{~~\llap{\textbullet}~~}
\usepackage{longtable}
\usepackage{array} % to call magic m
\usepackage{enumitem} % to remove vertical space for itemize
\usepackage{geometry}
\geometry{a4paper, landscape, margin=0.5in}
\newcommand{\tabitem}{~~\llap{\textbullet}~~}

\title{Litterature review by topic}
\author{Kenta Okuyama}

\begin{document}
\pagenumbering{gobble}
\maketitle
\tableofcontents
\newpage
\pagenumbering{arabic}

\section{Summary table of literature review}

%%%%%%%%%%%%%%%%%%%%%%%%%%%%%%%%%%%%%%% Table begin
\begin{longtable}[ht!]{ m{2cm} m{4cm} m{6cm} m{5cm} m{4cm} m{3cm} } %p{2cm}
	\hline \hline
\multicolumn{1}{c}{1.References} & \multicolumn{1}{c}{2.Objectives} & \multicolumn{1}{c}{3.Methods} & \multicolumn{1}{c}{4.Findings} & \multicolumn{1}{c}{5.Significance} & \multicolumn{1}{c}{6.Limitations} \\
	\hline
%%%%%%%%%%%%%%%%%%%%%%%%%%%%%%%%%%%%%%% Contents begin
%%%%%%%%% mason2018associations
	%%%1.Reference%%%
	Mason, 2018 \textit{Lancet Public Health} & 
	%%%2.Objectives%%%
	To examine the association between accessibility of fast food (FF) and physical activity (PA) faclities and adults' obesity. & 
	%%%3.Methods%%%
	\begin{itemize}[noitemsep,topsep=0pt]
		\item Cross-sectional  \item 22 UK biobank  \item 40-69 years old  \item Y: Waist, BMI, fat  \item X: PA 1000m, FF distance \item Z: Basic, deprivation, urbanization, residential density
	\end{itemize} &
	%%%4.Findings%%%
	\begin{itemize}[noitemsep,topsep=0pt]
		\item PA density was associated with lower adiposity \item FF proximity was asssociated with lower adiposity \item Robust association after stratifed by sex and income  
		\end{itemize} &
	%%%5.Significance%%%
		\begin{itemize}[noitemsep,topsep=0pt] \item Recreational facilities were found to be associated rather than urban designs \item Imply the change of facilities might be effective to lower the risk of obesity
		\end{itemize} &
	%%%6.Limitations%%%
		\begin{itemize}[noitemsep,topsep=0pt] \item No causality \item No mediator analysis
		\end{itemize} \\
	\hline

%%%%%%%%%%%%%%%%%%%%%%%%%%%%%%%%%%%%%%% 
%%%%%%%%% barrientos2017neighborhood
	%%%1.Reference%%%
	Barrientos, 2017 \textit{Am J Epidemiol} & 
	%%%2.Objectives%%%
	To examine the association between change of healthy food (HF) availablity, walkable environment, and recreational resources measured by subjectively (sub) and objectively (ob) and change of BMI among mid-older adults. &
	%%%3.Methods%%%
	\begin{itemize}[noitemsep,topsep=0pt] \item Longitudinal \item 6 US urban areas \item 45-84 years old \item Y: BMI \item X: HF availability and walkable environment (sub). Density of HF and Rec within 1 mile (ob) \item Z: Basic, residence duration (time-invariant). Marital status, cancer, income (time-varying) 
	\end{itemize} &
	%%%4.Findings%%%
	\begin{itemize}[noitemsep,topsep=0pt] \item Favorable change of both food and pa envrionment was associated with BMI reduction among obese/overweight people 
	\end{itemize} &
	%%%5.Significance%%%
	\begin{itemize}[noitemsep,topsep=0pt] \item First study taking into account for changes in environment measured by both subjectively and objectively against the change of BMI
	\end{itemize} &
	%%%6.Limitations%%%
	\begin{itemize}[noitemsep,topsep=0pt] \item Missing confoundings, ie. deprevation \item No mediation analysis
	\end{itemize} \\
	\hline

%%%%%%%%% hirsch2014change
	%%%1.Reference%%%
	Hirsch, 2014 \textit{Am J Epidemiol} &
	%%%2.Objectives%%%
	To examine the association between change of built envrionment and change of walking for recreation (rec) and transportatio (tran) among mid-older adults. &
	%%%3.Methods%%%
	\begin{itemize}[noitemsep,topsep=0pt] \item Longitudinal \item 6 US urban areas \item 45-84 years old \item Y: Self report walking time for rec and tran \item X: Population density, retail area, residential area, social destinations within 1 mile, distance to bus stop, and street network ratio \item Z: Basic, residence duration (time-invariant). Marital status, employment, income, car-ownership, self-rated health, arthritis (time-varying) 
	\end{itemize} &
	%%%4.Findings%%%
	\begin{itemize}[noitemsep,topsep=0pt] \item Higher residential use and a distance to bus stops were associated with a increase in walking \item Increases in te number of social destinations, walking destinations, and street connectivity were associated with greater increases in wallking for tran \item No changes in built environment measures were associated with leisure walking
	\end{itemize} &
	%%%5.Significance%%%
	\begin{itemize}[noitemsep,topsep=0pt] \item First study taking into account for changes in environment measured by GIS against the change of walking
	\end{itemize} &
	%%%6.Limitations%%%
	\begin{itemize}[noitemsep,topsep=0pt] \item Missing confoundings, ie. deprevation \item No mediation analysis
	\end{itemize} \\
	\hline

\end{longtable}


\newpage
\section{Neighborhood environments and obesity}
\subsection{References about physical environment}

\begin{itemize}
		\item {\bf Pearce (2009) -- -- Cross-sectional study}
				\begin{description}
						\item[Setting] New Zealand
						\item[Outcome]
						\item[Exposure]
						\item[Confounders]
						\item[Statistical analysis]
						\item[Significance] Examining the fast fod outlets and diet-related health based on previous finding: fast food outlets were stratified by neighborhood deprivation.
						\item[Conclusion]
    	         		\item[Takeaways] \mbox{}\\ 
    			        	\begin{itemize}
									\item[$\clubsuit$] People in low SES and living in high deprivided neighborhood tend to have worse diet-related health outcomes.
									\item[$\clubsuit$] In the US, higily deprived areas tend to have more fast food outlets.
									\item[$\clubsuit$] In the US, fast food outlets are associated with obesity, but it is mixed in other regions.
									\item[$\clubsuit$] High levels of residential segregation can be a reason that neighobrhoods in the US may influence individual-level health outcomes to a greater extent.  
									\item[$\clubsuit$] Targeted consumers in specific geographical localities, differences in land use planning strategies, and variations in resident's abilities to influence political decision making are caused by residential segregation.
									\item[$\clubsuit$] Future studies should include both health and unhealthy food outlets. 
							\end{itemize}
			    \end{description}

\newpage
		\item {\bf Kawakami (2011) -- -- Cross-sectional study}
				\begin{description}
						\item[Setting] Sweden 
						\item[Outcome]
						\item[Exposure]
						\item[Confounders]
						\item[Statistical analysis]
						\item[Significance] 
						\item[Conclusion]
    	         		\item[Takeaways] \mbox{}\\ 
    			        	\begin{itemize}
									\item[$\clubsuit$] 
									\item[$\clubsuit$] 
									\item[$\clubsuit$] 
									\item[$\clubsuit$] 
									\item[$\clubsuit$] 
									\item[$\clubsuit$] 
							\end{itemize}
			    \end{description}

\newpage
    \item {\bf Barrientos (2017)  -- 45-84 years old -- Prospective cohort}
		\begin{description}
			\item[Setting] 6 US urban areas
			\item[Outcome] BMI
			\item[Exposure] Perceptive: Healthy food availability, walkablle environment. -- survey was conducted to different sample and aggregated for 1 mile and used as proxy for this study samples. Objective: Density of supaermarkets and fruit-and-vegetable markets, recreational resouces within 1 mile form each participant.
			\item[Confounders] Time-invariant: age, sex, race, education, duration of residence. Time-varying: marital status, income, cancer diagnosis -- missing information was inputed from the closest year's examination.  
			\item[Statistical analysis] Within-person change for environment vs within-persion change in BMI was estimated by fixed-effects models. Model 1 included each measure separately -- food and physical activity (perceptive and objective), Model 2 included  percetive and objective simultaneously by food and physical activity, Model 3 included all at once, Model 4 included z-score aggregated by food and physical activity simultaneously. In regression, neighborhood measures were transformed into sd. Sensitivity analysis was made for non-movers during the study period.
    			\item[Significance] First longitudinal study accounting for changes in envrionments in both perceptive and objective measures again changes in BMI. 
    			\item[Conclusion] Favorable changes in both food and physical activity environment  was associated with BMI reductions in obese and overweight persons.
    			\item[Takeaways] \mbox{}\\ 
    				\begin{itemize}
    					\item[$\clubsuit$] In recent years, neighborhood study moved from a general framework to identifying the specific mechanism -- which neighborhood influence health.
    					\item[$\clubsuit$] Yet longitudinal evidences are still scarce.
    					\item[$\clubsuit$] The few studies investigated whether changes in food availability are related to changes in diet and BMI. Few studies for physical activity environment -- only 1 study found that improve of recreational facilities access was associated with decrease in BMI. 
				\end{itemize}
    		\end{description}

\newpage
	\item {\bf Mason (2018) -- 40-69 years old -- Cross sectional study}
		\begin{description}
			\item[Setting] 22 UK Biobank assessment centers
			\item[Outcome] Waist circumference, BMI and body fat percentage (BIA) -- centred around the mean  	
			\item[Exposure] Physical activity facilities density within 1000m. Distance to the nearest fastfood outlets categorized as 500, 500-999, 1000-1999, 2000m.
			\item[Confounders] Age, sex, ethnicity, education, income, employment, deprivation, urbanicity, residential density.
			\item[Statistical analysis] Multilevel multiple linear regression with random intercepts and coefficients, accounting for the nesting of individuals within assessment centers. -- final model controled for all covariates and non-exposure environment.
			\item[Significance] Large national dataset which 
			made it possible for sensitivity analysis to strengthen the robustness. Most previous studies focused on particular areas. Comprehensive confounding information from reliable dataset. Focused on commercial physical activity facilities as they are modifiable via regulatory.
    			\item[Conclusion] Physical activity facilities density was associated with lower adiposity. Fast food outlets proximity was also associated with lower adiposity. The association remained the same after stratified by sex, and income groups.
    			
			\item[Takeaways] \mbox{}\\ 
    				\begin{itemize}
    					\item[$\clubsuit$] Many research on access to fast-food outlets and obesity in the US, yet relatively little research on formal facilities for recreational physical activity.
    					\item[$\clubsuit$] Many research on walkability which focus more on urban designs than receattional facilities.
    					\item[$\clubsuit$] As food and physical activity environments are associated with one another, they are likely to confound the association with adiposity each other -- they should be included as one of the confounders. 
    					
    				\end{itemize}	
		\end{description}

\newpage
	\item{\bf Hirsch (2014) -- 45-84 years old -- Prospective cohort} 
		\begin{description}
			\item[Setting] 6 US urban cities
			\item[Outcome] Self report walking time for recreation and transportation.
			\item[Exposure] Population density, retail area, residential area, social destinations, walking destinations within 1 mile from each resident, distance to bus stops, and street network ratio.
			\item[Confounders] Time-invariate: age, sex, race, and education. Time-varying: income, employment, marital status, car ownership, self-rated health, and arthritis. 
			\item[Statistical analysis] Linear mixed models to estimate the associations of changes in the built environment and changes in walking (transportatin and recreation). Built environment measures were separetely included in the models to avoid multicolinearlity.
			\item[Significance] First study examined the time-varying GIS-based built environment measures and changes in walking.
			\item[Conclusion] Higher residential use and distance to busses were associated with a slightly increase in walking. Incrases in te number of social destinations, walking destinations, and street connectivity were associated with greater increases in wallking for transportation. Higher baseline levels of retail and walking destinations were associated with grater increases in leasure walking, but no changes in built environment measures were associated with leisure walking.
			\item[Takeaways] \mbox{}\\
				\begin{itemize}
					\item[$\clubsuit$] Samples completed at least 2 examinations including baseline, complete information on walking outcomes or built environment at examinations were included.
					\item[$\clubsuit$] Several movers vs non-movers studies found that residential relocation to more walkable environment resulted in increases in physical activity -- yet, these might contain unobservable preferences related to both choice of residential location and behavior.
					\item[$\clubsuit$] Future studies should aim to identify what types of changes are necessary to increase physical activity levels -- potential thresholds
				\end{itemize}
			
		\end{description}

\newpage
\item{\bf Sallis(1998) -- 7 studies  -- Review} 
		\begin{description}
			\item[Setting] NA 
			\item[Outcome] NA 
			\item[Exposure] NA 
			\item[Confounders] NA 
			\item[Statistical analysis] NA 
			\item[Significance] First study reviewed environmental and policy intervention to promote physical activity. Proposed model on how environmental and policy interventions are implemented.  
			\item[Conclusion] Implementing environment and political level interventions is difficult due to lack of conceptual models and difficulties of evaluation. Further research is needed, and multiple sectors should collaborate more. 
			\item[Takeaways] \mbox{}\\
				\begin{itemize}
					\item[$\clubsuit$] As of late 1900s, environmental and policy interventions were infrequently applied for the control of chronic diseases. 
					\item[$\clubsuit$] Ecological and social-ecological models of human behavior have evolved in the fields of sociology, psychology, education, and public health. 
					\item[$\clubsuit$] The concept of ecological model which describes the levels of influence on behaviors are developed by McLeroy. 
				\end{itemize} 
			
		\end{description}

\newpage
\item{\bf Ng(2014) -- 183 countries -- Systematic analysis} 
		\begin{description}
			\item[Setting] Worldwide 
			\item[Outcome] Prevalence of obesity (Overweight:25-30, Obesity:>=30) 
			\item[Exposure] NA 
			\item[Confounders] NA 
			\item[Statistical analysis] Mixed effects linear regression to correct for bias in self-reports. Spatiotemploral Gaussian process regression model to estimate prevalence with 95 percent uncertainty intervals. 
			\item[Significance] Up-to-date information for global obesity trends. It will be important for decision making on what action is needed and where progress is. 
			\item[Conclusion] Especially in low-income and middle-income countries, urgent intervention is needed to modify obesogenic environment. 
			\item[Takeaways] \mbox{}\\
				\begin{itemize}
					\item[$\clubsuit$] Although health risk of obesity is established and obesity prevalence has been increasing worldwide, no national sucess stories have been reported in the past 33 years. 
					\item[$\clubsuit$] The rising prevalence of overweight and obesity in several countries has been described as a global pandemic.  
					\item[$\clubsuit$] In 2010, overweight and obesity were estimated to cause 3-4 million deaths, 4 perent years of life lost, and 4 percent of disability-adjusted life-years worldwide. 
					\item[$\clubsuit$] Prevalence of obesity and overweight has incraesed substantially in the past 30 years. The pattern of increase differ by regions, and it has been attenuated in developed countries in the past 8 years.
					\item[$\clubsuit$] The increase of obesity prevalence can be explained by changes in energy intake such as high fat and calorie diet, decrease in energy expenditure such as decrease in physical activity, and chages in the gut microbiome.
					\item[$\clubsuit$] Most deaths atributable to overweight and obesity are cardiovascular deaths. Only 31 percent of he coronary heart disease risk and 8 percent of the stroke mortality risk associated with obesity is mediated through raised blood pressure and cholesterol.
				\end{itemize} 
			
		\end{description}
		
\newpage
\item{\bf Kawakami(2011) -- 35-80 years  -- Prospective cohort} 
		\begin{description}
			\item[Setting] Sweden
			\item[Outcome] Age-standardised incidence proportions (proportions of subjects who became cases among those who entered the study time interval) for men and women separetely. 
			\item[Exposure] Fastfood restaurants, bars/pubs, physical activity facilities, healthcare facilities by neighborhood count, individual buffer count, and distance. 
			\item[Confounders] Age, income, neighborhood deprivation index. 
			\item[Statistical analysis] Multilevel logistic regression for incidence propotions of CHD as an outcome. Model 1. CHD vs ne, Model 2. CHD vs ne + nedep, Model 3. CHD vs ne + nedep + age + income.  
			\item[Significance] First study conducted multilevel investigation to examine the longitudinal individual-level association between CHD and neighborhood availability of potentially health-promoting and health-damaging goods. 
			\item[Conclusion] 
			\item[Takeaways] \mbox{}\\
				\begin{itemize}
					\item[$\clubsuit$] Several studies found that neighborhood SES affecs cardiovascular health over individual-level SES.
					\item[$\clubsuit$] Living in deprived neighborhood tend to have limited access to healthy food resources, and it would lead to increase the risk of CHD. However, no clear pattern was found between the availability of different types of resources and level of neighborood deprivation. 
					\item[$\clubsuit$] There was a week association between neighborhood availability between CHD. There was an unexpected direction association between physical activity and healthcare facilities and CHD. 
					\item[$\clubsuit$] The association between neighborhood deprivation and CHD are well established, but the causal pathway is largely unknown. This study aimed to identify whether neighborhood health-damageing/promoting facililites lie in the causal pathway. The findings did not give expected results. Neighborhood deprivation is equal to neighborhood SES? Is the assocaition between neighborhood deprivation/ses and obesity also established? 
				\end{itemize} 
			
		\end{description}

\newpage
\item{\bf Hamano(2017) -- 0-14 years  -- Prospective cohort} 
		\begin{description}
			\item[Setting] Sweden
			\item[Outcome] Obesity diagnosed by ICD-10. 
			\item[Exposure] Fastfood restaurants, within SAMS and 1000m buffer.
			\item[Confounders] Age, maternal marital status,family income, parent education level, parent birth place, maternal urban/rural status, moving status, maternal age at hildbirth, parent hospitalisation, family history of obesity. 
			\item[Statistical analysis] Multilevel logistic regression for the cumulative rate of obesity. 
			\item[Significance] Multilevel analysis by controlling both individual and neighborhood SES with an large follow-up sample.
			\item[Conclusion] Fast food outlets were associated with childhood obesity after adjusting for individal and neighborhood SES.
			\item[Takeaways] \mbox{}\\
				\begin{itemize}
					\item[$\clubsuit$] Further studies should be done by taking other physical environmental features into accont.
				\end{itemize} 
			
		\end{description}

\newpage
\item{\bf Sundquist(2014) -- Unknown  -- Prospective cohort} 
		\begin{description}
			\item[Setting] Sweden - Stockholm 
			\item[Outcome] Clinically diagnosed type 2 diabetes.
			\item[Exposure] Neighborhood walkability index.  
			\item[Confounders] Age, gender, income, education, NDI. 
			\item[Statistical analysis] Multilevel logistic regression with individuals nested within their neighborhood. 
			\item[Significance] This was one of the few studies examine the objectively measured walkability and health oucome, i.e., incidence of diabetes in a large cohort. 
			\item[Conclusion] There was a siginficant association between walkable enviornment and incidence of type 2 diabetes after adjusting for neighborhood deprivation. However the association did not remain after further adjusting for individual socio-demographic characteristics. Future research should consider other potential risk factors for diabetes, such as traffic noise or air pollution which may come along with walkable environment. 
			\item[Takeaways] \mbox{}\\
				\begin{itemize}
					\item[$\clubsuit$] There is a need to examine whether individual and neighborhood socioeconomic characteristics may modify the association between the built environment and health related behaviors (Lovasi 2009). Because socioeconomically disadvantaged individuals may be less likely to benefit out from neighborhood envrionment. 
					\item[$\clubsuit$] No significant interaction was found between walkability and individual SES, or walkability and neighborhood deprivation.
					\item[$\clubsuit$] Several studies found association between walkability and physical activity after adjusting for individal and neighborhood socio-demographic factors (Sundquist 2011). 
				\end{itemize} 
			
		\end{description}

\newpage
\item{\bf Sundquist(2011) -- 20-65 years  -- Cross-sectional study}
		\begin{description}
			\item[Setting] Sweden - Stockholm 
			\item[Outcome] Physical activity: MVPA min/day, Time in 10-minutes bouts of MVPA min/day, Walking for AT min/day, Walking for leisure min/day. 
			\item[Exposure] Neighborhood walkability index.  
			\item[Confounders] Age, gender, marital status, family income, neighborhood income. 
			\item[Statistical analysis] Multilevel linear regression with individuals at the first level and neighborhoods at the second level. Model 1: PA vs walkbility. Model 2: PA vs walkability + individal characteristics + neiborhood income. 
			\item[Significance] First study in Sweden to examine objectively measured PA and obejectively measure walkability.  
			\item[Conclusion] There is a significant association between walkability and physical activity in Swedish adults. 
			\item[Takeaways] \mbox{}\\
				\begin{itemize}
					\item[$\clubsuit$] Interaction between walkability and neighborhood level SES wre not found.
					\item[$\clubsuit$] Australia had a significant interaction between SES and walkability (Leslie, 2007), while no interaction was found in Sweden and Belgium. It might be because of low SES inequalities in these countries.  
				\end{itemize} 
			
		\end{description}

\newpage
\item{\bf Gortmaker(2011) -- NA  -- Review} 
		\begin{description}
			\item[Setting] Australia, US 
			\item[Outcome] Effect of policy level intervention 
			\item[Exposure] NA
			\item[Confounders] NA
			\item[Statistical analysis] NA
			\item[Significance] Identify several cost-effective policies that gonvernment priotize for obesity prevention. 
			\item[Conclusion] A rapid increse of efforts for cost-effectiveness analysies of programmes and polices for obesity prevention is needed. 
			\item[Takeaways] \mbox{}\\
				\begin{itemize}
					\item[$\clubsuit$] Important causes for obesity is identified, which is a result of changes in the global food system - the movement from individual to mas preparation, producing more highly processed food.   
					\item[$\clubsuit$] Other factors: national wealth, government policy, cultural norms, built environment, genetic and epigenetic mechanisms, biological bases for food preferences, biological mechanisms that regulate motivation for physical activity amplify or attenumate the effect of thoes causes (nutrition change), and all infulence growth of the epidemic. 
					\item[$\clubsuit$] Most countries do not have enough monitoring data for population physical activity and diet pattern, and obesity prevalence, and it stands as barrier to set an appropriate goal and also assess progress. 
					\item[$\clubsuit$] Obesity could yeild in not only lowering future life expectancy but also increasing short-term and long-term healthcare spendings.
					\item[$\clubsuit$] Body weight response to a change of energy balance is slow. A small but chronic daily energy imbalance gap has caused the continuing weight gain seen in most countries.
					\item[$\clubsuit$] Population intervention for obesity should have effect on equity, acceptability to stakeholders, feasibility of implementation, affordability and sustainability to put policy decision forward. 
					\item[$\clubsuit$] Goverment is the main actor for obesity prevention.Obesity mainly burdens the health system, but various sectors, i.e. finance, eudcation, agriculture, transportation and urban planning have the greatest impact on creating environment conducive to prevention. 

				\end{itemize} 
			
		\end{description}

\newpage
\item{\bf Do(2018) -- 45-84 years  -- Longitudinal study}
		\begin{description}
			\item[Setting] US 6 cities
			\item[Outcome] BMI continuous in 1-5 exams over 12 years
			\item[Exposure] Spatial measure of neighborhood level racial/ethnic residential segregation
			\item[Confounders] Time-invariant: age, sex, nativity, education, duration of residence, language. Time-varying: working, marital status, income, smoking, cancer diagnosis, years since baseline.
			\item[Statistical analysis] Stratified analysis by race. Each specific segregation level was used for each race group.
			\item[Significance] Longitudinal analysis for racial/ethnic segregation and BMI. Examining further if segregation impact BMI independent of neighborhood poverty. This is important because neighborhood poverty is neighborhood deprivation - lack of healthy food outlets, physical activity facilities are often be a risk of obeisty, and understanding whether neighborhood poverty is a strong mediator? is critical for designing effective health policy interventions.
			\item[Conclusion] Neighborhood segregation is associated with both higher and lower BMI by race and certain circumstances.
			\item[Takeaways] \mbox{}\\
				\begin{itemize}
					\item[$\clubsuit$] Ratial residential segregation is consistently associated with poor health condition and high mortality.
					\item[$\clubsuit$] Black segregation is associated with incresed mortality risk and adverse birth outcome, while Hispanic segregation is associated with stronger social cohesion, and protective against dicrimination and adoption of harmful behaviors.
					\item[$\clubsuit$] Higher segregation was negatively associated with BMI for white females and positively associated for Hispanic females in cross-secitonal analysis. In longitudinal analysis, higher segregation was positively associated with BMI for black females, but negatively associated for Hispanic females and males.
				\end{itemize} 
		\end{description}

\newpage
\item{\bf Stafford(2007) -- 16-65+ years -- Cross-sectional study}
		\begin{description}
			\item[Setting] England and Scotland 
			\item[Outcome] BMI, Waist hip ratio 
			\item[Exposure] Crime, Policing, Physical dereliction, Highs street services, Leisure centres, Urban sprawl, Supermarkets, Fast-food outlets, Social disorder, Physical activity 
			\item[Confounders] Gender, Age, Occupation 
			\item[Statistical analysis] Structural Equasion model
			\item[Significance] First study to assess the causal path of modifiable neighborhood determinants to obesity.
			\item[Conclusion] Neighborhood disorders (e.g. crime, pilicing) and high street facilities were singnificantly associated with obesity. 
			\item[Takeaways] 
				\begin{itemize}
					\item[$\clubsuit$] Neighborhood deprivation is related to morbidity, although relationship is weaker and smaller in magnitude than the relationships between individual SES and morbidity.
					\item[$\clubsuit$] It is important to unpack the "black box" on why, what kind of modifiable factors lie in the causal path. 
				\end{itemize} 
		\end{description}

\newpage
\item{\bf Mckenbach(2014) -- NA -- Systematic review}
		\begin{description}
			\item[Setting] 
			\item[Outcome] 
			\item[Exposure] 
			\item[Confounders] 
			\item[Statistical analysis] 
			\item[Significance] Updated review of physical environment and obesity among adults focusing on high income countries. It distinguished the studies based on methodological quality, and made robust comparison.
			\item[Conclusion] Urban sprawl and land use mix were found to be associated with weight status only in North America. 
			\item[Takeaways] 
				\begin{itemize}
					\item[$\clubsuit$] Genetic factors underlie the propensity of individuals to become obese, but the rapid growth of obesity is hypothesized to be attributed from social and environmental causes. 
					\item[$\clubsuit$] There are heterogeneity of methods for measuring physical activity environments across continents. These should be addressed for future studies. 
				\end{itemize} 
		\end{description}

\newpage
\item{\bf Sobal(1989) -- NA -- Systematic review}
		\begin{description}
			\item[Setting] 
			\item[Outcome] 
			\item[Exposure] 
			\item[Confounders] 
			\item[Statistical analysis] 
			\item[Significance] First revirew of all studies examined the association between SES and obesity
			\item[Conclusion] In developed societies, women with lower SES had higher obesity prevalence. In developing societies, men, women and children with higher SES had higher obesity prevalence.
			\item[Takeaways] 
				\begin{itemize}
					\item[$\clubsuit$] Enviroments are recognized as possible factors on obesity since Price 1987, and Stunkard 1986.
					\item[$\clubsuit$] The possiblity that SES might be related to body weight was first raised by Veblen 1889, in his speculation that thinness became an ideal of feminine beauty in chivalric times
					\item[$\clubsuit$] Energy balance is established mechanism of obesity Grarrow 1978.
					\item[$\clubsuit$] Energy intake - solely from food consumption. Energy output - basal metabolism, physical activity, thermic effect of food.
					\item[$\clubsuit$] Low prevalence of obesity among low SES group in devloping countries might be due to lack of food or even to famine, coupled with high energy expenditure 
					\item[$\clubsuit$] High prevalence of obesity among high SES group in developng countries might be due to affordability to obtain adequte food, or cultural preference to fat body shapes. Obesity might be a sign of health and wealth in devloping countries, which is opposit from developed countries.
					\item[$\clubsuit$] The higher propensty to become obesity can be often observed among people who are from countries experienced significant food shortage. Their body were biologically mutated to store fat to survive.   
					\item[$\clubsuit$] There is a huge difference between male and female how they percieve obesity as negative or stigmatized image in developed countries. 


				\end{itemize} 
		\end{description}









\end{itemize}

\subsection{References about social environment}

\begin{itemize}
\item{\bf Coll-Planas(2016) -- >=60 years, 36 trials  -- Systematic review} 
		\begin{description}
			\item[Setting] NA 
			\item[Outcome] NA 
			\item[Exposure] NA
			\item[Confounders] NA
			\item[Statistical analysis] NA
			\item[Significance] First systematic review encompasses both mental and physical health of clinical interventions limited to randomized controll trial.
			\item[Conclusion] Social capital might promote general health of population, butmore high-quality research are needed.
			\item[Takeaways] \mbox{}\\
				\begin{itemize}
					\item[$\clubsuit$] Social capital is an umbrella concept - Structual (objective aspect): social networks, social contacts and participation belonging. Cognitive (subjective): social support, sense of belonging and trust. Itcan be further devided into - Bonding (ties in homogeenous group), Bridging (ties between hetrogeneous group), Linking (ties between people with unequal wealth, power and status).
				\end{itemize} 
		
		\end{description}
\newpage

\item{\bf Ehsan(2015) -- 39 studies  -- Systematic review} 
		\begin{description}
			\item[Setting] NA 
			\item[Outcome] NA 
			\item[Exposure] NA
			\item[Confounders] NA
			\item[Statistical analysis] NA
			\item[Significance] After 2005, more research have been done both cross-sectionally and longitudinally on the association between SC and mental health. This study updates the review of recent findings. 
			\item[Conclusion] Common mental disorders were associeted with individual cognitive SC. 
			\item[Takeaways] \mbox{}\\
				\begin{itemize}
					\item[$\clubsuit$] The association between social determinants, such as low-ses, unemploynment, low educational attainment and negative health concequences are well established. - Social capital can extend the concept of social deteminants by including quantity and quality of social relationships.
					\item[$\clubsuit$] The association between "cognitive" SC and mental health was found in te systematic review by De Silva et al. 2005. 

				\end{itemize} 
		
		\end{description}

\newpage
\item{\bf McNeill(2006) --   -- Systematic review} 
		\begin{description}
			\item[Setting] NA 
			\item[Outcome] NA 
			\item[Exposure] NA
			\item[Confounders] NA
			\item[Statistical analysis] NA
			\item[Significance] Conceptualizing social envioronment by defining each component related to physical activity. Inconsistent terminology is often used for defining social environment, so conceptualizing is critical to facilitate further high quality studies.
			\item[Conclusion] 
			\item[Takeaways] \mbox{}\\
				\begin{itemize}
					\item[$\clubsuit$] Environmental and individual factors shape simultaneously behavior based on social-cognitive theory.
					\item[$\clubsuit$]

				\end{itemize} 
		\end{description}

\end{itemize}


\section{Neighborhood environment and longitudinal change of BMI}
\begin{itemize}
	\item{\bf Rachele(2017) -- 40-65 years -- longitudinal study} 
	
		\begin{description}
			\item[Setting] Brisbane, Australia
			\item[Outcome] BMI (self-reported)
			\item[Exposure] Index of Relative Socioeconomic Disadvantage for each census tract.
			\item[Confounders] age, education, occupation, income, neighborhood self-selection.
			\item[Statistical analysis] 3-level mixed-effects linear regression models. Observations varied overtime within individuals = level1, observations clustered within individuals who were specified = level2, individuals clustered within neighborhood = level3. Participants who returned to the study after a nonresponse and had moed were included. All variables were  observed at each wave (0, 1, 2, 3).
			\item[Significance] Longitudinal model taking into account the change of neighborhood enviornment.
			\item[Conclusion] It may not be causal for the association between neighborhood socioeconomic disadvantage and BMI among middle aged and older adults.
			\item[Takeaways] \mbox{}\\
				\begin{itemize}
					\item[$\clubsuit$] The between-individual coefficients show that women living in more disadvantaged neighborhoods had, on average, a higher BMI, but not for men. 
					\item[$\clubsuit$] The within-individual coefficients show that moving to a new neighborhood with a 1-quintile increase in disadvantage was not associated with a within-individual increase in BMI.
				\end{itemize}
		\end{description}
\end{itemize}


\begin{itemize}
	\item{\bf WHO(2014) -- -- report} 
		\begin{description}
			\item[Setting] NA 
			\item[Outcome] NA 
			\item[Exposure] NA
			\item[Confounders] NA
			\item[Statistical analysis] NA
			\item[Significance] NA
			\item[Conclusion] 
			\item[Takeaways] \mbox{}\\
				\begin{itemize}
					\item[$\clubsuit$] 
				\end{itemize}
		\end{description}
\end{itemize}


\begin{itemize}
	\item{\bf Global BMI Mortality Collaboration(2016) -- -- meta analysis} 
		\begin{description}
			\item[Setting] NA 
			\item[Outcome] NA 
			\item[Exposure] NA
			\item[Confounders] NA
			\item[Statistical analysis] NA
			\item[Significance] NA
			\item[Conclusion] 
			\item[Takeaways] \mbox{}\\
				\begin{itemize}
					\item[$\clubsuit$] 
				\end{itemize}
		\end{description}
\end{itemize}



\end{document}
